\documentclass[utf8,bachelor]{gradu3}
\usepackage{graphicx} % kuvien mukaan ottamista varten
\usepackage{booktabs} % hyvä kauniiden taulukoiden tekemiseen
% HUOM! Tämän tulee olla viimeinen \usepackage koko dokumentissa!
\usepackage[bookmarksopen,bookmarksnumbered,linktocpage]{hyperref}
\addbibresource{eetu_rantakangas_kandi.bib} % Lähdetietokannan tiedostonimi
\begin{document}
\title{Innovatiiviset opetusmenetelmät}
\translatedtitle{Innovative teaching methods}
\studyline{Kaikki suuntautumisvaihtoehdot}
\avainsanat{Opetusmenetelmä, innovaatio}
\keywords{Teaching method, innovation}
\tiivistelma{%
  Tämä kirjoitelma on esimerkki siitä, kuinka
  {gradu3}-tutkielmapohjaa käytetään.  Se sisältää myös
  käyttöohjeet ja tutkielman rakennetta koskevia ohjeita.

  Tutkielman tiivistelmä on tyypillisesti lyhyt esitys, jossa
  kerrotaan tutkielman taustoista, tavoitteesta, tutkimusmenetelmistä,
  saavutetuista tuloksista, tulosten tulkinnasta ja johtopäätöksistä.
  Tiivistelmän tulee olla niin lyhyt, että se, englanninkielinen
  abstrakti ja muut metatiedot mahtuvat kaikki samalle sivulle.
}
\abstract{%
  This document is a sample {gradu3} thesis document class
  document.  It also functions as a user manual and supplies
  guidelines for structuring a thesis document.

  The English abstract of a thesis should usually say exactly the same
  things as the Finnish tiivistelmä.
}

\author{Eetu Rantakangas}
\contactinformation{\texttt{eetu.rantakangas@iki.fi}}
% jos useita tekijöitä, anna useampi \author-komento
\supervisor{-}
% jos useita ohjaajia, anna useampi \supervisor-komento

\maketitle

\preface
Tähän voit kirjoittaa tutkielmasi esipuheen.  Tutkielmissa on
harvemmin esipuheita, mutta jos sen kirjoitat, pidä se lyhyenä
(enintään sivu).

Esipuheen tulisi kertoa ennemminkin tutkielmaprosessista kuin
tutkielman sisällöstä.  Esimerkiksi jos tutkielman aiheen valintaan
tai tekemiseen liittyy jokin erikoinen sattumus, voit siitä kertoa
esipuheessa.  Tapana esipuheessa on myös kiittää nimeltä mainiten
tärkeimpiä tutkielman tekemisessä auttaneita ihmisiä -- ainakin
ohjaajia, puolisoa ja lapsia.  (Yleensä perhe on auttanut vähintään
tukemalla ja kannustamalla.)

Esipuhe kannattaa kirjoittaa minä-muodossa. Tavanomaista on myös
allekirjoittaa se.

Jyväskylässä \today

\bigskip

Tutkielman tekijä

\begin{thetermlist}
\item[\TeX] Donald Knuthin 1977--1989 laatima eräajotyyppinen
  ladontajärjestelmä \parencite[ks.][]{knuth86:_texbook}. 
\item[\LaTeX] \TeX in \parencite[ks.][]{knuth86:_texbook} päälle
  rakennettu rakenteisten kirjoitelmien ladontaan tarkoitettu
  järjestelmä \parencite[ks.][]{lamport94:_latex}.  Siitä on nykyään
  käytössä versio \LaTeXe.
\end{thetermlist}

\mainmatter

\chapter{Johdanto}

Tutkimusongelma: Nykymuotoisen koulun uudistamisen tarpeesta
//TODO Tähän viittauksia erilaisiin koulujen uudistamistarveuutisiin sekä tutkimuksiin siitä kuinka koulut vastaavat nyky-yhteiskunnan tarpeita. Myös viittauksia kouluviihtyvyyteen
//Opetuskokonaisuudet muuttumassa laajemmiksi ainerajat kaatumassa

Suomen koulut ovat maailman huippua \parencite[][]{koulutHuippua}, mutta se ei tarkoita sitä, etteikö niitä voisi entisestään parantaa. Nykykouluissa oppilaat tuntevat olonsa kuormittuneiksi. \parencite [][]{oppilaidenKuormitus} Lisäksi oppilaiden kouluviihtyvyys on huonoa. Tähän liittyy myös esimerkiksi koulukiusaaminen. \parencite[][]{vakivalta} Opettajien tyytymättömyyttä taas lisää esimerkiksi palkattoman työn määrä. \parencite[][]{palkatonTyo}

Tutkimus keskittyy esittelemään erilaisia innovatiivisiksi miellettäviä opetusmenetelmiä sekä esittelemään sitä, miksi ne ovat innovatiivisia, ja kuinka nämä innovaatiot voivat auttaa opetuksen uudistuspyrkimyksissä. Lopuksi hahmotellaan suurempaa innovatiivisten menetelmien mahdollistamaa “perinteisen luokkahuonetilanteen” muutosta.

Suomi pärjää hyvin Pisa-vertailuissa, mutta kouluissa viihtyminen ei ole huipputasoa \parencite[][]{kouluViihtyvyys}, ja oppilaiden kuormituksen tunne kouluissa on yleistynyt. \parencite [][]{oppilaidenKuormitus} Tämän takia onkin seuraavaksi syytä kääntää katseet siihen, että miten viihtyvyyttä voitaisiin parantaa ja kuormituksen tunnetta vähentää. Esitellyt tavat eivät välttämättä ole suoria ratkaisuja siihen, ja ne ovat lähinnä yksittäisiä case-esimerkkejä. Niistä voi kuitenkin tulla merkittävä osa koulumaailman uudistustarpeen ratkaisukokonaisuutta. Ne auttavat hahmottamaan kaikkia niitä uusia tapoja ja mahdollisuuksia, joiden suuntaan toimintaa voidaan viedä. 

Voidaan myös kysyä, että mitä apua on uusista toimintatavoista, jos oppilaat kokevat koulutyön kuormittavaksi. Jos toimintatapoja tulee vain lisää, ja vanhat säilyvät rinnalla, koulunkäynnistä tulee helposti vaikeammin ennustettavaa ja opintojen työtapojen yhtäaikainen määrä voi olla omiaan aiheuttamaan hämmennystä.

Kuormittavuuden kannalta olisikin tärkeää, että uusia opetusmenetelmiä valittaessa huomioitaisiin niiden kuormittavuutta ja toisaalta myös kehitettäisiin ratkaisuja kuormittavuuteen. Jos oppilas esimerkiksi näkee läksyt ja deadlinet kerralla yhdestä paikkaa, voi hän helposti vain alkaa käydä listaa läpi ja merkkaamaan jo tehdyt, sen sijaan että muistelisi mitä kaikkea pitääkään tehdä ja musertuisi sekavan ja suuren läksykasan alle. Kun hahmottaa mitä pitää tehdä, ja voi merkata asian tehdyksi, niin työnteko on paljon mukavampaa. Myös läksyjen kerääntymistä voi pyrkiä välttämään samanlaisilla ratkaisuilla.

\chapter{Teoriaosa}
\section{Opetusmenetelmien määritelmä}
Innovaatiolla tarkoitetaan "idean tai keksinnön muuttamista tuotteeksi tai palveluksi josta asiakkaat ovat valmiita maksamaan." \parencite[][]{innovaatio} Toisin sanoen innovaatio tuottaa oleellista lisäarvoa. Oleellista lisäarvoa opetuksessa voi olla esimerkiksi oppilaan nopeampi oppiminen, opettajan työtehtävien automatisoituminen tai keveneminen, opettamisen tai oppimisen muuttaminen mielekkäämmäksi, varsinaisen asian oppimisen lisäksi sen rinnalla opittavat uudet taidot jne.


\section{Innovatiivisten opetusmenetelmien käyttöönoton vaatimukset}
Uusien ratkaisujen tuominen opetukseen ja koulumaailmaan ei ole täysin mutkatonta. Singaporessa koulujärjestelmää kehitetään tehokkaasti ja järjestelmällisesti. Singaporen malli mahdollistaa konkreettiset pitkän tähtäimen strategiat. Suomessa on enemmän hallinnon tasoja, mutta resursointi hallintoon on vähäisempää. Singaporessa esimerkiksi tietokoneet tarjotaan valtion puolelta kilpailutettuna, mutta kouluilla on myös mahdollisuus tehdä omat hankinnat, jos valtion tarjoamat vaihtoehdot eivät tyydytä tarpeita. Ylätason hallinto tukee esimerkiksi tällä tavalla kehittämistä voimakkaammin kuin Suomessa. \parencite[][]{koulunArki} 

Opettajat kokevat ajanpuutteen suurimmaksi esteeksi tietotekniikan hyödyntämiselle koulussa. Toisena syynä on koulun puutteellinen laitteisto ja kolmantena se, ettei oppilailla ole laitteita käytössä vapaa-ajallaan. \parencite[][]{sites}

//TODO Mikseivät opettajat käytä enemmän tietotekniikkaa
//TODO Opetusta uudistavat opettajat

Tutkielman varsinainen teksti alkaa aina luvulla ''Johdanto''.  Sen
kirjoittamisen voi hyvin jättää aivan tutkielman kirjoitusprosessin
loppuvaiheisiin.

Johdanto kannattaa aloittaa napakasti esittämällä heti alussa
tutkielman pääväite tai tutkimuskysymys.  Tämän jälkeen kannattaa
selventää asioita määrittelemällä tarvittavat
käsitteet.\footnote{Määritelmät vasta väitteen jälkeen! Äläkä
  jaarittele johdannossa.}  Johdannossa voit myös kertoa, miksi väite
on käytännön tai tieteen (tai parhaimmillaan molempien) kannalta
relevantti ja mielenkiintoinen.  Erinomaista olisi, jos kertoisit
johdannossa lyhyesti myös, mikä on tutkielmasi kontribuutio eli mitä
sellaista tietoa tutkielmasi sisältää, jonka olet itse selvittänyt sen
sijaan että olisit sen lähteestä lukenut.  Kontribuutio voi hyvin olla
myös se, että olet itse tarkastanut jonkin lähteestä löytyneen
väitteen todenperäisyyden.  Johdannon lopuksi on tapana esitellä
lyhyesti tutkielman rakenne -- mitä missäkin luvussa käsitellään.

Tämä malli käsittelee Jyväskylän yliopiston tietotekniikan laitoksella
tehtävien kandidaatintutkielmien ja pro gradu "=töiden laatimista
avustavaa \LaTeX-kirjoitelmaluokkaa gradu3 (versio \graduclsversion).
Apua sen käyttämiseen voit saada Tutkielma-TeX"-postituslistalta
(\url{http://lists.jyu.fi/mailman/listinfo/tutkielma-tex}).
Kommentteja, parannusehdotuksia ja bugiraportteja voit lähettää myös
minulle suoraan.

\chapter{Tutkielman rakenne}

Yhteensä tutkielmassa on hyvä olla 5--9 numeroitua
lukua, siis Johdanto ja Yhteenveto mukaan lukien.  Tarvittaessa voit
käyttää alilukuja tarkempaan jäsentelyyn.

Johdannon ja Yhteenvedon väliin jääviä lukuja kutsutaan toisinaan
tutkielman \textit{käsittelyosaksi}.  Usein sen katsotaan jakaantuvan
vielä kahtia, jolloin käsittelyosa alkaa \textit{teoriaosalla} ja
päättyy joko \textit{päälauseeseen}, \textit{konstruktiiviseen osaan}
tai \textit{empiiriseen osaan}.

\section{Teoriaosa}

Tutkielman teoriaosan tarkoituksena on esitellä tutkielmassa
tarvittava teoreettinen tausta.  Tämä on syytä tehdä vähintään sillä
tarkkuudella, että tutkielman lukija pystyy pelkästään tutkielman
itsensä perusteella ymmärtämään kaikki tutkielmassa käytettävät
erityiskäsitteet ja "=menetelmät.  Hyvässä tutkielmassa on myös
perusteltu (vaihtoehdot kirjallisuudesta esille tuoden), miksi juuri
nämä käsitteet ja menetelmät on työssä käytössä.

Teoriataustan järkevä esitys- ja käyttötapa riippuu siitä, minkä
tyyppisestä tutkimuksesta tutkielmassasi on kyse.
Matemaattis-teoreettisen työn teoriaosa on aivan eri näköinen kuin
konstruktiivisen ohjelmistonkehitystyön teoriaosa; näistä myös eroaa
olennaisesti ihmistieteellisiin traditioihin nojautuvan määrällisen
tai laadullisen tutkimustuön teoriaosa.  Muita samantyyppisiä
tutkielmia ja julkaistuja tutkimusraportteja lukemalla saat kyllä
käsityksen siitä, mitä omalta työltäsi vaaditaan.

\section{Teorian jälkeen}

Teoriaosan jälkeen tulee työsi varsinainen kontribuutio:
\begin{itemize}
\item Matemaattis-teoreettisessa työssä se on yleensä jono itse
  laatimiasi määritelmiä ja lemmoja, jotka kulminoituvat työn
  päälauseen todistukseen.
\item Konstruktiivisessa työssä se on itse laatimasi tietokoneohjelma
  tai muu artefakti.
\item Empiirisessä työssä se on jotain empiiristä tutkimusmenetelmää
  soveltamalla saavutettu joukko empiirisiä tuloksia.
\end{itemize}

Tutkielmassa kontribuutio esitellään varsin tarkasti, tehdyt valinnat
perustellen.  Erityisesti matemaattis-teoreettisissa ja empiirisissä
töissä on syytä noudattaa kulloisenkin tutkimustradition käytänteitä
-- esimerkiksi ihmistieteellinen koeasetelma on kuvattava tarkasti.

\chapter{Lähteiden käyttö}

Teoriaosa perustuu lähes aina yksinomaan lähdekirjallisuuteen.  Myös
kontribuutio"-osassa on lähteiden käyttö toisinaan tarpeen.

Muista varoa plagiointia. Jos kopioit joko sellaisenaan tai lievästi
muutettuna (tai esimerkiksi englannista suomennettuna) tekstiä jostain
lähteestä, tee selväksi, että olet tehnyt niin.  Merkitse lainaukset
(lainausmerkeillä tai muulla selkeällä tavalla) ja anna täsmällinen
lähdeviite.  Jos et lainaa sanatarkasti, merkitse tekemäsi muutokset.
Useimmissa tilanteissa on kuitenkin parempi esittää asia omin sanoin,
mieluiten useamman lähteen perusteella.  Merkitse tällöinkin
käyttämäsi lähteet.

Lähdeluettelon laadintaan {gradu3} käyttää automaattisesti
\textsc{Bib\LaTeX}-järjestelmää \parencite{biblatex-manual} ja sen
Chicago-tyyliä \parencite{biblatex-chicago-manual}.  Tämän
automatiikan saa pois \string\documentclass-optiolla manualbib, mutta
tällöin joudut itse huolehtimaan lähdeluettelon muotoilusta eivätkä
tässä luvussa esitetyt tekniikat ole (välttämättä) käytettävissä.
Huomaa, että Tietotekniikan laitoksen graduissa on suositeltavaa
käyttää Chicago-tyylistä lähdeluetteloa.

\section{Lähdeviittaukset}

Lähteisiin voit viitata kahdella tavalla.  Ensinnäkin voit käyttää
lähdettä lauseen subjektina: \textcite[luku~8.8.4]{aho-compilers}
kuvaavat lyhyesti graafinvärityksen käyttämisen kääntäjän
rekisterinvalinnassa.  Tällöin viittaukseen käytetään
\string\textcite-komentoa.  Toisekseen lähdeviite voi olla
sivuhuomautus, jota ei ääneen luettaessa mainita: Graafinväritys on
yksi mahdollinen tapa valita
rekisterit \parencite[luku~8.8.4]{aho-compilers}.  Tämä toteutetaan
\string\parencite-komennolla.

Sekä \string\textcite- että \string\parencite-komennot ottavat kolme
parametria, joista kaksi on valinnaisia.  Ensimmäinen (valinnainen)
parametri on esihuomautus, toinen (valinnainen) parametri on
jälkihuomautus ja kolmas (pakollinen) parametri on lähdeviittauksen
koodi \parencite[ks.][luku~3.7]{biblatex-manual}.  Edellisen virkkeen
lähdeviite tehtiin seuraavalla komennolla:

\begingroup\footnotesize
\begin{verbatim}
\parencite[ks.][luku~3.7]{biblatex-manual}
\end{verbatim}
\endgroup

Jos komennolle annetaan vain yksi valinnainen (eli hakasulkeisiin
kirjoitettu) argumentti, se tulkitaan jälkihuomautukseksi.  Jos
halutaan antaa vain esihuomautus ilman jälkihuomautusta, on
jälkihuomautus jätettävä tyhjäksi
\parencite[ks.][]{biblatex-manual}:

\begingroup\footnotesize
\begin{verbatim}
\parencite[ks.][]{biblatex-manual}
\end{verbatim}
\endgroup

On myös mahdollista viitata useampaan lähteeseen samassa viittauksessa 
%
\parencites%
  [ks.][luku~3.7]{biblatex-manual}%
  [ks.~lähteiden käytöstä yleisesti myös][luku~5.3.2]%
    {biblatex-chicago-manual}%
\relax.
%
Tämä tehdään komennolla \string\parencites, jolle annetaan kutakin
lähdettä kohti samat argumentit kuin yksittäiselle
\string\parencite"-komennolle.  Komento on hyvä (mutta ei pakko)
päättää \string\relax-komentoon, jotta yllätyksiltä vältyttäisiin.

\begingroup\footnotesize
\begin{verbatim}
\parencites%
  [ks.][luku~3.7]{biblatex-manual}%
  [ks.~lähteiden käytöstä yleisesti myös][luku~5.3.2]%
    {biblatex-chicago-manual}%
\relax.
\end{verbatim}
\endgroup

Jos jaat \string\parencites"-komennon usealle riville, päätä rivit
kommenttimerkillä (kuten yllä), jotta tulokseen ei ilmaantuisi
ylimääräisiä välilyöntejä.

\section{Lähdetietokanta}

Lähteet lisätään erilliseen \textsc{Bib\TeX}"-tiedostomuodossa olevaan
lähdetietokantaan.  Sen laatimisessa voit käyttää apuna monia
lähteidenhallintajärjestelmiä, mutta sen voi laatia myös käsin.
Tietokannan nimi kirjoitetaan \string\addbibresource-komennon
argumentiksi.

\textsc{Bib\TeX}-muotoinen lähdetietokanta on erityisellä tavalla
muotoiltu tekstitiedosto.  Se koostuu tietueista, jotka alkavat
@-merkillä ja sitä seuraavalla tietuetyypin nimellä.  Muu osa
tietueesta kirjoitetaan aaltosulkeiden sisään.  Esimerkiksi edellä
mainittu kääntäjäkirja \parencite{aho-compilers} voidaan
esittää seuraavanlaisena tietueena:

\begingroup\footnotesize
\begin{verbatim}
@Book{aho-compilers,
  author =       {Alfred V. Aho and Monica S. Lam and Ravi Sethi and
                  Jeffrey D. Ullman},
  title =        {Compilers},
  subtitle =     {Principles, Techniques, \& Tools},
  publisher =    {Pearson Addison Wesley},
  year =         2007,
  address =      {Boston},
  edition =      2
}
\end{verbatim}
\endgroup%

Tämän tietueen tyyppi on book, joka tarkoittaa luonnollisestikin
kirjaa.  Aaltosulkeiden sisällä oleva ensimmäinen sana on tietueen
koodi, jota käytetään \string\textcite- ja
\string\parencite"-komennoissa.  Sen jälkeen tulee pilkku ja joukko
nimettyjä kenttiä kuten kirjan kirjoittaja (author), nimi (title),
alaotsikko (subtitle) ja julkaisija (publisher).  Kenttien sisällöt
laitetaan aaltosulkeisiin, tosin pelkkiä numeroita sisältävät kentät
voi kirjoittaa ilmankin.

Kirjoittajien nimet kirjoitetaan tietuekenttään pääosin täysin
tavanomaisella tavalla.  Vaihtoehtoisesti nimi voidaan esittää myös
muodossa sukunimi-pilkku-etunimi (Aho, Alfred V.), ja joissakin
erityistapauksissa (esimerkiksi moniosainen väliviivaton sukunimi) se
on myös pakko tehdä niin.  Jos kirjoittajia on useita, heidän nimensä
erotetaan sanalla and (jota ei pidä suomentaa!).  Jos kaikkia
kirjoittajia ei luetella, laitetaan viimeisen nimen perään (ilman
lainausmerkkejä) ''and others''.

Jos lähteen tekijäksi on merkitty jokin organisaatio, sen nimi pitää
kirjoittaa ylimääräisiin
aaltosulkeisiin \parencite[esim.][]{unicode620}:

\begingroup\footnotesize
\begin{verbatim}
@Book{unicode620,
  author =       {{Unicode Consortium}},
  title =        {The Unicode Standard, Version 6.2.0},
  year =         {2012},
  url =          {http://www.unicode.org/versions/Unicode6.2.0/},
  urldate =      {2013-01-29}
}
\end{verbatim}
\endgroup

Jos lähteellä ei jostain syystä ole lainkaan mimettyä tekijää, tulee
author-kenttä jättää kokonaan pois, jolloin lähdeviitteeseen tulee
tekijän tilalle otsikko \parencite[esim.][]{presidential-novel}:

\begingroup\footnotesize
\begin{verbatim}
@Book{presidential-novel,
  title =        {O},
  subtitle =     {A Presidential Novel},
  publisher =    {Simon \& Schuster},
  year =         {2011},
}
\end{verbatim}
\endgroup

Tieteellinen lehtiartikkeli \parencite[esim.][]{strachey-fundamentals}
kirjoitetaan esimerkiksi seuraavanlaiseksi tietueeksi:

\begingroup\footnotesize
\begin{verbatim}
@Article{strachey-fundamentals,
  author =       {Christopher Strachey},
  title =        {Fundamental Concepts in Programming Languages},
  journal =      {Higher-Order and Symbolic Computation},
  year =         2000,
  volume =       13,
  number =       {1--2},
  pages =        {11--49},
  doi =          {10.1023/A:1010000313106}
}
\end{verbatim}
\endgroup

Huomaa erityisesti kenttä doi, johon voi kirjoittaa artikkelin
digitaalisen tunnisteen (Digital Object Identifier, DOI).  Se on
yleensä parempi valinta kuin mikään URL, koska DOI on pysyvä
artikkelin tunnistetieto.  Useimmat DOIt on lisäksi muutettavissa
URLiksi lisäämällä sen alkuun \url{http://dx.doi.org/}.

Jos netissä olevan lähteen DOI ei ole tiedossa (tai sitä ei ole
lainkaan), voi käyttää url-kenttää ja sen kaverina urldate-kenttää,
jolla ilmaistaan (muodossa VVVV--KK--PP) verkossa olevan lähteen
viittauspäivä.  Linkki kannattaa valita huolella siten, että se on
mahdollisimman tarkka ja mahdollisimman pitkään voimassa -- jos
sivulla on erikseen osoitettu pysyvä linkki (engl.~\emph{permanent
  link}), sitä on syytä käyttää.

Viitattaessa WWW-sivuun, joka ei ole kirja tai artikkeli tai muukaan
julkaisu, voidaan käyttää
online-tietuetyyppiä \parencite[esim.][]{debian-social-contract}:

\begingroup\footnotesize
\begin{verbatim}
@Online{debian-social-contract,
  title =        {Debian Social Contract},
  year =         {2004},
  url =          {http://www.debian.org/social_contract.en.html},
  urldate =      {2013-01-29}
}
\end{verbatim}
\endgroup

Jotkin lähteet ovat toimitettuja kokoomateoksia, jotka koostuvat
itsenäisistä artikkeleista.  Yleensä tällöin viitataan johonkin sen
osa"-artikkeliin \parencite[esim.][]{prechelt-credibility} eikä koko
kokoomateokseen.  Tällöin sekä teos että viitatut artikkelit lisätään
tietokantaan omina tietueinaan, ja kussakin artikkelitietueessa
viitataan kokoomateokseen käyttäen
crossref"-kenttää:\footnote{Sallittua on myös yhdistää artikkeli ja
  kokoomateos yhdeksi InCollection-tietueeksi, esimerkiksi jos
  kokoomateoksesta viitataan vain yhteen artikkeeliin.  Tällöin
  kokoomateoksen nimi tulee booktitle"-kenttään eikä crossref"-kenttää
  käytetä.}

\begingroup\footnotesize
\begin{verbatim}
@Collection{making-software,
  editor =       {Andy Oram and Greg Wilson},
  title =        {Making Software},
  subtitle =     {What Really Works, and Why We Believe It},
  publisher =    {O'Reilly},
  year =         2011
}
@InCollection{prechelt-credibility,
  author =       {Lutz Prechelt and Marian Petre},
  title =        {Credibility, or Why Should I Insist on Being
                  Convinced},
  crossref =     {making-software},
  pages =        {17--34}
}
\end{verbatim}
\endgroup

Huomaa, kuinka kokoomateoksella on toimittajia (editor) eikä tekijöitä
(author).

Tarkempia tietoja lähdetietokannan rakenteesta löytyy
\textsc{Bib\TeX}in manuaalista \parencite{bibtexing},
\textsc{Bib\LaTeX}in manuaalista \parencite[luku~2]{biblatex-manual}
sekä \textsc{Bib\LaTeX}-Chicagon manuaalista
\parencite[luvut 5.1--5.2]{biblatex-chicago-manual}.  Lisää
esimerkkejä löydät myös tämän oppaan lähdekoodista.

\section{Lähdeluettelo}

Lähdetietokanta muutetaan lähdeluetteloksi apuohjelmalla {biber}.  Se
on varsin uusi, joten se puuttuu useimmista koneista, joiden
\TeX-asennus ei ole aivan ajantasalla.  Yliopiston suorakäyttökoneista
se löytyy tällä hetkellä vain charra.it.jyu.fi-koneesta.
Ubuntu-asennuksiin se on saatavissa versiosta 12.10 alkaen ja
Debian-asennuksiin Wheezystä alkaen.  Windowsiin se on asennettavissa
MikTeX-pakettina miktex-biber-bin.

Komentoriviltä biberin käyttö on yksinkertaista.  Kun \LaTeX\ (tai
pdf\LaTeX) on kerran ajettu, ajetaan biber parametrinaan dokumentin
nimi.  Tämän jälkeen ajetaan \LaTeX\ (tai pdf\LaTeX) vähintään kerran
(kunnes edellisen ajon lopussa ei enää pyydetä uutta ajoa).
Esimerkiksi näin:

\begingroup\footnotesize
\begin{verbatim}
$ pdflatex malliopas
[...]
Package biblatex Warning: Please (re)run Biber on the file:
(biblatex)                malliopas
(biblatex)                and rerun LaTeX afterwards.
[..]
Output written on malliopas.pdf (18 pages, 96855 bytes).
Transcript written on malliopas.log.
$ biber malliopas
INFO - This is Biber 0.9.9
[...]
INFO - Output to malliopas.bbl
$ pdflatex malliopas
[...]
LaTeX Warning: Label(s) may have changed. Rerun to get cross-references right.
[...]
Output written on malliopas.pdf (21 pages, 107373 bytes).
Transcript written on malliopas.log.
$ pdflatex malliopas
[...]
Output written on malliopas.pdf (21 pages, 107509 bytes).
Transcript written on malliopas.log.
\end{verbatim}
\endgroup

\section{Tiedossa olevat ongelmat}

Lähdeluettelon ja lähdeviitteiden toiminta ei ole toistaiseksi aivan
virheetöntä.

Jos artikkelilla ei ole tekijää, lähdeluettelossa kyseisen artikkelin
merkintä alkaa vuosiluvulla.  Tähän ei ole toistaiseksi tiedossa
korjausta.

Jos lähdetietokantaan kirjoittaa urldate-päiväyksen, tulee se
lähdeluetteloon virheellisessä muodossa.  Tämä vika on korjattu
\textsc{Bib\LaTeX}-Chicagon versiossa 0.9.9b (julkaistu 6.12.2012).

\chapter{Tutkielmapohjan erityispiirteet}

Pääsääntöisesti {gradu3} käyttäytyy kuten \LaTeX in mukana
tuleva {report}-kirjoitelmaluokka.  Eroja kuitenkin on:
\begin{itemize}
\item Sinun ei tarvitse ladata {inputenc}-, {fontenc}-
  eikä {babel}-pakettia.
  \begin{itemize}
  \item Käyttämäsi merkistö sinun pitää ilmoittaa
    {\string\documentclass}-komennon optiona.  Nykyään {utf8} on
    yleensä sopiva valinta, joskin joissakin tilanteissa latin1 tai
    latin9 voi tulla myös kyseeseen.
  \item Jos tutkielmasi on englanninkielinen, ilmoita se
    {\string\documentclass}-komennon optiolla {english}.
  \end{itemize}
\item Jos tutkielmasi on kandidaatintutkielma, käytä
  {\string\documentclass}-komennon optiota {bachelor}.
\item Ilmoita tutkielmasi metatiedot taulukossa~\ref{tbl:metatiedot}
  esitetyillä komennoilla.  Ne tulee antaa ennen
  {\string\maketitle}-komentoa.
\begin{table}[h]\centering
  \begin{tabular}{lp{9cm}}
    \toprule
    Komento & Tarkoitus \\
    \midrule
    {\string\title}
    & Työn otsikko (älä käytä {\string\thanks}-komentoa) \\
    {\string\translatedtitle}
    & Suomenkielisen työn englanninkielinen otsikko,
    englanninkielisen työn suomenkielinen otsikko\\
    {\string\studyline}
    & Suuntautumisvaihtoehtosi \\
    {\string\tiivistelma}
    & Suomenkielinen tiivistelmä \\
    {\string\abstract}
    & Englanninkielinen abstrakti \\
    {\string\avainsanat}
    & Suomenkieliset avainsanat \\
    {\string\keywords}
    & Englanninkieliset avainsanat \\
    {\string\author}
    & Kirjoittajan nimi (jos useita, anna kukin omana komentonaan -- {\string\and}-komentoa ei tueta) \\
    {\string\contactinformation}
    & Kirjoittajan yhteystiedot \\
    {\string\supervisor}
    & Tutkielman ohjaaja (jos useita, anna kukin omana komentonaan)\\
    \bottomrule
  \end{tabular}
  \caption{Metatietojen ilmoituskomennot}\label{tbl:metatiedot}
\end{table}
\item Voit \string\maketitle-komennon jälkeen halutessasi kirjoittaa
  esipuheen.  Sen otsikon saat komennolla \string\preface.
\item Mahdollisen esipuheen jälkeen voit kirjoittaa termiluettelon
  käyttämällä thetermlist-ympäristöä.  Sen sisällä voit käyttää
  \string\item[\textit{termi}]"-komentoa merkitsemään määriteltävän
  termin.
\item Käytä \string\maketitle-komennon ja mahdollisten esipuheen ja
  termiluettelon jälkeen \string\mainmatter"-komentoa.  Se laatii
  automaattisesti tarvittavat sisällys-, kuvio- ja taulukkoluettelot.
\item Komentoja \string\subsubsection, \string\paragraph{} ja
  \string\subparagraph{} ei tueta.
\item Liitteet eivät ole lukuja (\string\chapter) vaan alilukuja
  (\string\section).
\item Lähdeluettelon ja lähdeviitteiden tekemisestä kerrottiin
  edellisessä luvussa.
\end{itemize}

\chapter{Yhteenveto}

Tutkielman viimeinen luku on Yhteenveto.  Sen on hyvä olla lyhyt;
siinä todetaan, mitä tutkielmassa esitetyn nojalla voidaan sanoa
johdannon väitteen totuudesta tai tutkimuskysymyksen vastauksesta.
Yhteenvedossa tuodaan myös esille tutkielman heikkoudet (erityisesti
tekijät, jotka heikentävät tutkielman tulosten luotettavuutta), ellei
niitä ole jo aiemmin tuotu esiin esimerkiksi Pohdinta-luvussa.  Tässä
luvussa voidaan myös tuoda esille, mitä tutkimusta olisi tämän
tutkielman tulosten valossa syytä tehdä seuraavaksi.

Jos Yhteenveto alkaa pitkittyä, se kannattaa jakaa kahtia niin, että
tulosten tulkinta otetaan omaksi Pohdinta-luvukseen, jolloin
Yhteenvedosta tulee varsin lyhyt ja lakoninen.

Yhteenvedon jälkeen tulee \string\printbibliography-komennolla
laadittu lähdeluettelo ja sen jälkeen mahdolliset liitteet.

\printbibliography

\appendix
\section{Siirtyminen gradu2:sta gradu3:een}

Keskeneräisen tutkielman siirtäminen gradu2:sta gradu3:een ei ole
kovin vaikeata.  Aluksi on totta kai vaihdettava
\string\documentclass-komennossa gradu2 gradu3:ksi.  Komennon
optioista suurin osa on poistettava, koska niitä ei enää tueta;
ainoastaan merkistön ilmoittava optio jää jäljelle.  Mahdollinen
kandi-optio vaihdetaan optioksi bachelor.

Taulukossa~\ref{tbl:cmdchange} on lueteltu tarvittavat
komentovaihdokset.  Viiva tarkoittaa, ettei vastaavaa komentoa ole
lainkaan.  Huomaa erityisesti uudet komennot.

\begin{table}[h]\centering
  \begin{tabular}{ll}
    \toprule
    gradu2                 & gradu3  \\
    \midrule
    ---                    & \string\maketitle \\
    ---                    & \string\supervisor \\
    \string\acmccs         & --- \\
    \string\aine           & \string\subject\\
    \string\copyrightowner & --- \\
    \string\fulltitle      & --- \\
    \string\laitos         & \string\department\\
    \string\license        & --- \\
    \string\linja          & \string\studyline\\
    \string\paikka         & --- \\
    \string\setauthor      & \string\author\\
    \string\termlist       & thetermlist-ympäristö\\
    \string\tyyppi         & \string\type\\
    \string\yhteystiedot   & \string\contactinformation\\
    \string\yliopisto      & \string\university\\
    \string\ysa            & --- \\
    \bottomrule
  \end{tabular}
  \caption{Komentomuutokset gradu2:sta gradu3:een}
  \label{tbl:cmdchange}
\end{table}

Isoin työ voi aiheutua lähdeluettelon laatimistekniikan muuttumiseen
sopeutumisesta.

\section{Harvemmin tarvittavat ominaisuudet}

Aiemmin esiteltyjen lisäksi gradu3 tarjoaa seuraavat lisäominaisuudet:
\begin{itemize}
\item \LaTeXe:n vakio-optiot draft ja final toimivat.
\item Vaikka tutkielman suomenkielisyyttä ei tarvitse erikseen
  mainita, finnish-optio toimii.
\item \string\university-komennolla voit ilmoittaa tutkielman
  kotiyliopistoksi jonkin muun kuin Jyväskylän yliopiston.
\item  \string\department-komennolla voit ilmoittaa tutkielman
  kotilaitokseksi jonkin muun kuin Tietotekniikan laitoksen.
\item \string\subject-komennolla voit ilmoittaa tutkielman
  oppiaineeksi jonkin muun kuin tietotekniikan.  Huomaa, että oppiaine
  tulisi suomenkielisissä tutkielmissa kirjoittaa genetiivimuodossa ja
  isolla alkukirjaimella (''Tietotekniikan''), englanninkielisissä
  tuktkielmissa in-preposition kanssa (''in Information Technology'').
\item \string\type-komennolla voit ilmoittaa tutkielman tyypin, jos se
  on jokin muu kuin pro gradu (oletus) tai kandidaatintutkielma
  (optiolla bachelor).
\item \string\setdate-komennolla voit asettaa päivämäärän
  haluamaksesi.  Anna komennolle kolme parametria -- päivä,
  kuukausi ja vuosi -- numeerisessa muodossa.
\item Ympäristöllä chapterquote voit laittaa luvun alkuun
  mietelauseen.  Sillä on yksi pakollinen parametri (lainauksen
  attribuutio).
\item Komento \string\graduclsdate\ sisältää käytössä olevan gradu3:n
  julkaisupäivämäärän ja \string\graduclsversion\ sen versionumeron.
\end{itemize}

%\begin{thebibliography}
% \bibitem{koulutHuippua}
%Kupari, Välijärvi, Andersson, Arffman, Nissinen, Puhakka, Vettenranta \textit{Pisa 2012 ensituloksia}, 
%Opetus- ja kulttuuriministeriön julkaisuja 2013:20
%Gummerus, Jyväskylä 2014.
%\end{thebibliography}






\end{document}
