\documentclass[utf8,bachelor]{gradu3}

\usepackage{graphicx} % kuvien mukaan ottamista varten
\usepackage{booktabs} % hyvä kauniiden taulukoiden tekemiseen
% HUOM! Tämän tulee olla viimeinen \usepackage koko dokumentissa!
\usepackage[bookmarksopen,bookmarksnumbered,linktocpage]{hyperref}

\addbibresource{eetu_rantakangas_kandi.bib} % Lähdetietokannan tiedostonimi
\begin{document}
\title{Innovatiiviset opetusmenetelmät}
\translatedtitle{Innovative teaching methods}
\studyline{Kaikki suuntautumisvaihtoehdot}
\avainsanat{Opetusmenetelmä, innovaatio}
\keywords{Education method, innovation}
\tiivistelma{
  %Tutkielman tiivistelmä on tyypillisesti lyhyt esitys, jossa kerrotaan tutkielman taustoista, tavoitteesta, tutkimusmenetelmistä, saavutetuista tuloksista, tulosten tulkinnasta ja johtopäätöksistä. Tiivistelmän tulee olla niin lyhyt, että se, englanninkielinen abstrakti ja muut metatiedot mahtuvat kaikki samalle sivulle.

  Suomen koulut ovat maailman huippua. Voisivatko innovatiiviset opetusmenetelmät auttaa huipulla pysymisessä ja ennen kaikkea, mitkä olisivat parhaat käytännöt niiden kehittämiseen ja hyödyntämiseen? Tutkielmassa käydään läpi joitain innovatiivisia opetusmenetelmiä sekä aiheeseen liittyvää tutkimusta. Tuloksissa esitellään hyviä käytäntöjä innovatiivisten opetusmenetelmien kehittämiseen ja hyödyntämiseen.
}
\abstract{
  Finnish schools are the best in the World. Could innovative educational methods, ease staying on the top and what would be the best practices for developing and make good use of those? Thesis look through some innovative educational methods and studies about the topic. Results present good practices about developing innovative educational methods and making good use of those.
}

\author{Eetu Rantakangas}
\contactinformation{\texttt{eetu.rantakangas@iki.fi}}
% jos useita tekijöitä, anna useampi \author-komento
\supervisor{Sanna Mönkölä}
% jos useita ohjaajia, anna useampi \supervisor-komento

\maketitle

\mainmatter

\chapter{Johdanto}

% Tutkimusongelma: Nykymuotoisen koulun uudistamisen tarpeesta
% TODO Tähän viittauksia erilaisiin koulujen uudistamistarveuutisiin sekä tutkimuksiin siitä kuinka koulut vastaavat nyky-yhteiskunnan tarpeita. Myös viittauksia kouluviihtyvyyteen
% Opetuskokonaisuudet muuttumassa laajemmiksi ainerajat kaatumassa
%Lopuksi hahmotellaan suurempaa innovatiivisten menetelmien mahdollistamaa “perinteisen luokkahuonetilanteen” muutosta.

%Suomen koulut ovat maailman huippua ~\parencite[][]{koulutHuippua}, mutta se ei tarkoita sitä, etteikö niitä voisi entisestään parantaa. Tutkielma käy läpi erilaisia innovatiivisiksi miellettäviä opetusmenetelmiä sekä esittelee sitä, miksi ne ovat innovatiivisia, ja kuinka nämä innovaatiot voivat auttaa opetuksen uudistuspyrkimyksissä.

Suomi pärjää hyvin Pisa-vertailuissa~\parencite[][]{koulutHuippua}, mutta kouluissa viihtyminen ei ole huipputasoa~\parencite[][]{kouluViihtyvyys}. Myös oppilaiden kuormituksen tunne kouluissa on yleistynyt~\parencite [][]{oppilaidenKuormitus} ja oppilaiden viihtyvyyttä vähentää lisäksi koulukiusaaminen~\parencite[][]{vakivalta}. Opettajien tyytymättömyyttä taas lisää esimerkiksi palkattoman työn määrä~\parencite[][]{palkatonTyo}. Tämän takia onkin seuraavaksi syytä kääntää katseet siihen, miten viihtyvyyttä voitaisiin parantaa ja kuormituksen tunnetta vähentää. Esitellyt innovaatiot eivät välttämättä ole suoria ratkaisuja siihen, mutta niistä voi kuitenkin tulla merkittävä osa koulumaailman uudistustarpeen ratkaisukokonaisuutta. Ne auttavat hahmottamaan kaikkia niitä uusia ja erilaisia mahdollisuuksia, joiden suuntaan toimintaa voidaan viedä. 

Voidaan myös kysyä, että mitä apua on uusista toimintatavoista, jos oppilaat kokevat koulutyön kuormittavaksi. Jos toimintatapoja tulee vain lisää, ja vanhat säilyvät rinnalla, koulunkäynnistä tulee helposti vaikeammin ennustettavaa ja opintojen työtapojen yhtäaikainen määrä voi olla omiaan aiheuttamaan hämmennystä. Kuormittavuuden kannalta olisikin tärkeää, että uusia opetusmenetelmiä valittaessa huomioitaisiin niiden kuormittavuutta ja kehitettäisiin ratkaisuja kuormittavuuteen. Jos oppilas esimerkiksi näkee läksyt ja deadlinet kerralla yhdestä paikasta, voi hän helposti vain alkaa käydä listaa läpi ja merkkaamaan jo tehdyt, sen sijaan että muistelisi mitä kaikkea pitääkään tehdä ja musertuisi sekavan ja suuren läksykasan alle. Kun hahmottaa mitä pitää tehdä, ja voi merkitä asian tehdyksi, niin työnteko on paljon mukavampaa. Läksyjenkin kerääntymistä voi pyrkiä välttämään samankaltaisilla ratkaisuilla. 

%On myös esitetty näkemyksiä, että ainerajoja pitäisi madaltaa esimerkiksi lukioissa teemaopinnoilla, ja tietotekniset ratkaisut voivat auttaa myös tässä tavoitteessa. http://www.hs.fi/kotimaa/a1410505971531 

%TODO Tarkenna ainerajojen madaltamista. Kirjoita koulujen säästöpaineista.

%\subsection{Tutkimusongelma}
Teknologian kehityksen myötä uudet opetusmenetelmät ovat joka tapauksessa rantautumassa kouluihin sähköistyvien ylioppilaskirjoitusten vauhdittamana. Onkin tarkoituksenmukaista tarkastella erilaisia tapoja, joilla teknologiaa voidaan hyödyntää. Tutkielmassa pyritään antamaan vastaus kysymykseen, että mitkä ovat parhaat käytännöt innovatiivisten opetusmenetelmien kehitykseen sekä käyttöönottoon. Tämä on erityisen tärkeää siksi, että koulun arkea saataisiin kevennettyä sen sijaan, että uudet välineet loisivat lisää rasitteita henkilökunnalle ja oppilaille.

%Tutkimuksessa pyrin selvittämään pystytäänkö koulun arkea keventämään innovatiivisten opetusmenetelmien avulla. Lisäksi pyrin antamaan suosituksia niiden kehittämiseen ja käyttöönottoon.


%\subsection{Tutkielman sisältö}


\chapter{Mitä innovatiiviset opetusmenetelmät ovat ja miten ne otetaan käyttöön?} %TODO Muuta nimi kuvaavammaksi
Innovatiivisten opetusmenetelmien kehittämisessä ja käyttöönotossa avainroolissa suomalaisessa koulujärjestelmässä ovat opettajat. Koulujen käytännöt ovat oikeastaan ainoa opettajien vapautta opettaa haluamallaan tavalla rajoittava tekijä. Opetussuunnitelma asettaa myös raamit sisällöille, mutta jättää kuitenkin huomattavan määrän vapautta toteuttaa suunnitelmaa halutuilla menetelmillä ja työtavoilla. Luvussa käydään läpi sekä opetusmenetelmien määritelmää, että sitä miten eri tahot voivat toimillaan innovatiivisten menetelmien käyttöönottoa edistää.


\section{Innovatiivisten opetusmenetelmien määritelmä}
Innovaatiolla tarkoitetaan ''idean tai keksinnön muuttamista tuotteeksi tai palveluksi josta asiakkaat ovat valmiita maksamaan.'' ~\parencite[][]{innovaatio} Toisin sanoen innovaatio tuottaa oleellista lisäarvoa. Oleellista lisäarvoa opetuksessa voi olla esimerkiksi oppilaan nopeampi oppiminen, opettajan työtehtävien automatisoituminen tai keveneminen, opettamisen tai oppimisen muuttaminen mielekkäämmäksi ja varsinaisen asian oppimisen lisäksi sen rinnalla opittavat uudet taidot.

Tutkimuksessa tarkasteltavat opetusmenetelmät on rajattu innovatiivisiin ohjelmisto- ja laitteistoratkaisuihin. Kuitenkin myös esimerkiksi opetusmetodeissa voi tapahtuma muutoksia tietotekniikan avulla. Uusia lisäarvoa tuovia metodeita pystytään kehittämään myös täysin ilman tietotekniikkaa. Esimerkiksi flipped classroom \ref{flippedClass} voi toimia periaatteessa ilman tietotekniikkaakin. Tietotekniikkaa hyödynnettäessä oppilaat voivat kotona katsoa Youtube-videon seuraavan tunnin aiheesta ja tunnille tullessaan alkaa tehdä suoraan tehtäviä ilman opettajan johdanto-osuutta~\parencite[][]{flipped}. Toisaalta olisi aivan mahdollista että oppilaat lukisivat johdanto-osuuden perinteisestä kirjasta. Myös esimerkiksi luokkahuoneen rakenne voidaan muuttaa perinteisestä pulpettimallisesta joksikin muuksi, jolloin voidaan käyttää erilaisia työtapoja kuin perinteisessä luokkahuoneessa.

\section{Innovatiivisten opetusmenetelmien käyttöönoton vaatimukset}
Uusien opetusmenetelmien käyttöönotossa täytyy huomioida sekä tekniikka että ihmiset ja molempien yhteispeli. Lisäksi hallinto- ja päätöstentekotapa ovat tärkeässä roolissa. Jos osa-alueista joku jätetään huomiotta, niin se on omiaan aiheuttamaan muutosvastarintaa ja käytännön ongelmia. Tietotekniikan opetuskäyttöön liittyvistä ongelmista on kerrottu ansiokkaasti Jussi Tuukkasen pro gradu tutkielmassa~\parencite[][s. 5-30]{kemia}.

%TODO lyhytalalukujen esittely/yhteenveto, pidennä
\subsection{Hallinnolliset ratkaisut}
Uusien ratkaisujen tuominen opetukseen ja koulumaailmaan ei ole täysin mutkatonta ja johdon rooli voi siinä olla merkittävä. Suomessa johto on pirstaloitunut useille tasoille ja koulujen autonomia on suhteellisen vahvaa, joten ennen kaikkea tämä on kiinni rehtoreista. Esimerkiksi Singaporessa koulujärjestelmää kehitetään ministeriövetoisesti tehokkaasti ja järjestelmällisesti. Singaporen malli mahdollistaa myös konkreettiset pitkän tähtäimen strategiat. Suomessa on enemmän hallinnon tasoja, mutta toisaalta resursointi hallintoon on vähäisempää. Singaporessa esimerkiksi tietokoneet tarjotaan valtion puolelta kilpailutettuna, mutta kouluilla on myös mahdollisuus tehdä omat hankinnat, jos valtion tarjoamat vaihtoehdot eivät tyydytä tarpeita. Ylätason hallinto tukee esimerkiksi tällä tavalla kehittämistä voimakkaammin kuin Suomessa.~\parencite[][]{koulunArki}

Rehtorien rooli nouseekin uusien opetusmenetelmien käyttöönotossa merkittäväksi. Etenkin, kun rehtoreilla on keskimäärin muita opettajia paremmat valmiudet tietotekniikan rutiinikäyttöön~\parencite[][]{itviikkoWilen}. Rehtori pystyy vaikuttamaan innovatiivisten opetusmenetelmien käyttöönottoon erityisesti kannustamalla opettajia ja näyttämällä esimerkkiä. Johto pystyy myös vaikuttamaan mahdollisuuksiin opettaa eri tavoilla esimerkiksi lisäämällä kaksoistuntien määrää, jolloin opettajat voivat viedä oppilaat ulos tai tehdä muita aikaa enemmän vaativia asioita. Välillä suuria oppilasryhmiä voidaan laittaa esimerkiksi katsomaan elokuva. Näin pystytään järjestämään osalle opettajista aikaa valmistella sellaisia tietotekniikka-avusteisia tunteja, joiden valmistelu on ensimmäistä kertaa järjestettäessä työläämpää kuin tavallisten tuntien.

Lisäksi kannattaa järjestää koulutusta laitteisiin ja ohjelmistoihin. Samalla voi valita koulutuksessa parhaiten pärjänneen osuuden henkilökunnasta toimimaan jatkossa muiden apuna laitteiden ja ohjelmistojen käytössä. Rehtorien panos innovatiivisten opettajien taitojen levittämisessä muille opettajille voisikin olla merkittävä, toisaalta ainakaan vielä vuonna 2006 rehtorit eivät nähneet tietotekniikkaa erityisen merkittävänä osana opetusta ja sen kehittämistä~\parencite[][]{koulunArki}.

\subsection{Opetusta uudistavat opettajat}
Kouluihin uusia opetuskäytänteitä vie opettajien ''innovatiivinen osajoukko''. Tämän opettajaosuuden aktivoiminen ja heidän taitojensa siirtäminen muille koulun opettajille voisikin olla merkittävässä roolissa uusien opetusmenetelmien jalkauttamisessa kouluihin. Pääosin innovatiiviset opettajat eivät eroa muista opettajista ja toimivat samoin kuin muutkin. Kuitenkin oppilaslähtöisten menetelmien sekä tietotekniikan käyttö erottaa heitä muista. He tuulettavat koulua tuomalla sinne joitain uusia asioita. Usein he eivät saa tukea uusille toimintatavoilleen muulta yhteisöltä. Olisikin tärkeää, että yhteisö sekä rehtorit pystyvät tukemaan heitä, ja auttamaan uusien käytänteiden leviämisessä. ~\parencite[][]{koulunArki}

Yleisesti ottaen opettajat kokevat ajanpuutteen suurimmaksi esteeksi tietotekniikan hyödyntämiselle koulussa. Toisena syynä tulee koulun puutteellinen laitteisto ja kolmantena se, ettei oppilailla ole laitteita käytössä vapaa-ajallaan. ~\parencite[][]{sites} Myös kannustimet tietotekniikan käyttöön puuttuvat~\parencite[][]{itviikkoWilen}. Ontariossa Kanadassa asiaa on lähdetty ratkaisemaan yhteisöllisyydellä.
%\subsection{Yhteistyön lisääminen}
Siellä on saatu nostettua alueen koulujen oppimistuloksia lisäämällä opettajien välistä yhteistyötä. Kun opettajilla on hyvät keskusteluyhteydet kollegoiden kanssa, ongelmat ratkeavat nopeammin ja tehokkaammin kuin yksin pähkäillessä. Lisäksi kouluissa on sitouduttu yhteisiin periaatteisiin:

\textit{“Ontarion koulut ovat sitoutuneet muutamiin yhteisiin periaatteisiin. Ne ovat oppilaslähtöisyys, avoimuus, läpinäkyvyys, luovat opetusmetodit sekä oppilaiden rohkaiseminen ja kannustaminen niin, että epäonnistumisetkin nähdään oppimisena. Yhteiset tavoitteet ovat esillä luokkien seinillä.”} ~\parencite[][]{KanadanMalli}

Jos oppilaat voidaan sitouttaa koulun sääntöihin, kuten tupakointikieltoon, miksei opettajiakin voitaisi sitouttaa yhteisiin tavoitteisiin? Opetuksen yhteistyötä saadaan lisättyä esimerkiksi aineiden yhteisillä tunneilla. Tietotekniikkaa voidaan hyödyntää maantiedon tunnilla, tai liikuntaa yhdistää historiantuntiin esimerkiksi GPS-pohjaisilla ratkaisuilla~\parencite[][]{heijoe}. Tietysti myös esimerkiksi biologian ja fysiikan yhteneviä kohtia pystytään hyödyntämään. Oppimateriaalien jako opettajien kesken on myös nykyään paljon helpompaa kuin ennen. Kollegat pystyvät vähentämään toistensa työtaakkaa suunnittelemalla materiaaleja yhdessä, jakamalla niitä keskenään sekä hyödyntämällä jo netistä valmiiksi löytyvää materiaalia.

%yhteen edellisen kanssa. huom. käsitellään laajemmin esittely/käytäntöosassa
%\subsection{Oppimateriaalien jakaminen ja verkosta löytyvien hyödyntäminen}

%todo viittaus kemian graduun
%\subsection{Opetuksen pedagoginen tvt-tuki}
Myös opetuksen pedagogisella tieto- ja viestintätekniikan tuella pystytään mahdollisesti viemään tietotekniikan käyttöä osaksi koulun arkea. Pedagogisella tvt-tuella tarkoitetaan tukea, joka ei auta opettajaa niinkään teknisesti, vaan pääasiassa käyttämään opetuksessaan ja liittämään opetukseensa tieto- ja viestintäteknologiaa hyödyntävää materiaalia. Yksi tapa antaa tätä tukea on tietotekniikan opettajan tarjoama pedagoginen tvt-tuki ja toinen taas muiden opettajien tarjoama.

Muiden opettajien tarjoama tuki saadaan aikaiseksi esimerkiksi kouluttamalla kaikki opettajat käyttämään tietoteknistä ratkaisua tai välinettä. Näistä opettajista perataan ne, jotka omaksuivat käytön parhaiten, eli noin 1/10. Tämän opettajajoukon tehtävänä on jatkossa tukea muita tvt-välineiden käytössä. Toimintatapa sopii esimerkiksi aktiivitaulun tai muiden laitteiden käytön opetteluun varsin hyvin, mutta pitkällä aikavälillä ei välttämättä auta opettajia omaksumaan uusia välineitä ja ratkaisuja. Niistä paras asiantuntemus on yleensä tietotekniikan opettajalla tai koulun ulkopuolisilla asiantuntijoilla. Joka tapauksessa vertaisopetus on toteutettavissa koulun henkilöstömäärää lisäämättä, joten muiden opettajien vertaistuen hyödyntäminen on erittäin suositeltavaa. Vertaisilta tulevan tuen kohdalla myös kommunikointi toimii paremmin ja muutosvastarintaa syntyy vähemmän. ~\parencite[][]{kemia, koulunArki}

\chapter{Innovatiiviset menetelmät}
%TODO pohjustusta
Tässä luvussa käydään läpi innovatiivisia opetusmenetelmiä ja nostetaan niistä esiin piirteitä, jotka ovat omiaan tuomaan lisäarvoa opetukseen ja oppimiseen. Varsinaisten opetusmenetelmien lisäksi innovatiivisia ratkaisuja löytyy esimerkiksi laitteisto- ja ohjelmistoratkaisuista.
%Luku on jaettu alalukuihin: laitteistot, ohjelmistot ja menetelmät

\section{Menetelmät} \label{flippedClass}
%\subsection{Materiaalin ja kokemusten luokittelu ja jakaminen}
Opetushallituksen sivuilta löytyy hyvät käytännöt -osio, jossa on listattuna monenmoisia opettajien opetuksessaan hyödyntämiä ideoita~\parencite[][]{hyvatkaytannot}. Kuka tahansa opettaja pystyy lähettämään sivustolle omia ideoitaan. Tällaiset kokoavat sivustot nousevatkin yhä tärkeämpään asemaan materiaalin määrän koko ajan kasvaessa. Samantapaisia kokoavia sivustoja on myös esimerkiksi opetusmateriaalien ja pelien löytämiseen ja jakamiseen. Esimerkiksi graphite.orgista löytyy tageihin perustuvaa sähköisen oppimateriaalin jaottelua. Sivusto lemill.net puolestaan tarjoaa mahdollisuuden jakaa omaa tai muiden oppimateriaalia sekä merkitä materiaalin käyttöoikeudet.
% http://www.graphite.org
% http://lemill.net
% http://www.peda.net/veraja/konnevesi/lukio/ophhanke2010/pelit

%\subsection{Käänteinen opetus}
%sido paremmin
Matematiikan puolella~\parencite[][]{maot} ehkä eniten puhetta viime aikoina herättänyt opetusmenetelmä on ns. flipped classroom eli käänteinen opetus~\parencite[][]{flipped}. Oppilaille ei opeteta asioita koulussa, vaan oppilaat opettelevat asiat kotona, ja koulussa keskitytään laskemiseen. Jos jotain ei ole ymmärretty, niin opettaja neuvoo. Tällä tavoin opettajalle jää enemmän aikaa oppilaiden henkilökohtaiseen ohjaukseen kuin mallissa, jossa opettaja aina esittelisi luokan edessä uuden asian. Vastaavasti opettajan on mahdollista ottaa koko ryhmän kanssa esiin vain kaikkein haastavimmiksi osoittautuneet asiat. Käytännössä jokainen opiskelija opiskelee omaan tahtiinsa, ja opettaja auttaa tarvittaessa. Tämä myös rikkoo perinteisiä kurssi- ja vuosiluokkarajoja ja opiskelijoiden oma tahti jatkuu läpi koko koulun. Tavoitteena kaikilla on kuitenkin täyttää opetussuunnitelman asettamat vaatimukset. Samalla saadaan hyöty irti mahdollisista sähköisistä ohjelmista, jotka antavat sopivantasoista laskettavaa ja korjaavat heti jos vastaus meni väärin.


% http://www.katsomo.fi/?progId=378128

% http://yle.fi/elavaarkisto/artikkelit/tietokoneet_koulujen_opetuksen_tukena_105423.html#media=105426




%TODO muuta tätä
\section{Oppimispelit ja sovellukset}
%Oppimispelit ja sovellukset pystytään jakamaan kahteen pääluokkaan: ensisijaisesti viihdyttävät ja ensisijaisesti opettavat. Jälkimmäisten hyödyllisyyttä pystyy ehkä parhaiten arvioimaan viihdyttävyyden määrällä ja oppimisen tehokkuudella. Jos sovelluksen käyttö on vähintään yhtä viihdyttävää kuin tunnilla olo, ja siitä oppii yhtä hyvin tai paremmin asiat kuin perinteisellä oppitunnilla, sovellus on yksinkertaisesti hyvä.

%\subsection{esimerkki pelillistämisestä ja tehokkuudesta}
Hyvä esimerkki tehokkaasta oppimissovelluksesta on Memrise~\parencite[][]{memrise}. Sivuston tarkoitus on opettaa kielten sanastoja. Perinteiseen sanakirjaan verrattuna sivustossa on valtavasti etuja. Se esimerkiksi tietää, kuinka nopeasti ja millä tavoin opit parhaiten. Tämä perustuu käyttäjädatan hyödyntämiseen - tiedetään, kuinka monta kertaa sanaa pitää toistaa, että ihmiset sen keskimäärin oppivat. Lisäksi tiedetään, että kuinka monta kertaa kannattaa antaa tehtävä, jossa kysytään: “Mikä sana on näistä neljästä?”, ja kuinka monta kertaa kannattaa kysyä, että: “Miten sana kirjoitetaan?”. Etuna sanakirjaan on myös se, että sivusto tarjoaa sanoihin liittyviä muistisääntöjä kuvina ja lauseina sekä antaa sanoista ääntämisesimerkit. Nämä helpottavat muistamista entisestään, koska oppiminen on monikanavaisempaa. Lisäksi sivusto pyytää “kastelemaan” opitut sanat tietyin väliajoin, etteivät ne unohdu.

Sivustolla on myös jonkin verran pelillisyyttä. Eniten kyseistä kielikurssia viikon aikana käyneet näkyvät ranking-listoilla kymmenen kärkenä ja sanojen opiskelussa käytetään kukkavertauksia. Ensin kukka istutetaan, jonka jälkeen kerätään sato. Tässä vaiheessa sana on opittu. Tämän jälkeen täytyy kukkaa vielä kastella, jotta se ei kuole, eli opittu ei unohdu. 

Memrisestäkin toisaalta puuttuu kaikki opittua soveltava. Sieltä ei löydy lukemista joka liittyisi juuri opittuihin sanoihin tai esimerkiksi mahdollisuutta tuottaa tekstiä tai keskustella. Myöskään kielioppia ei sieltä opi. Näiden asioiden opettaminen jääkin opettajan ja perinteisen kieltenopetuksen vastuulle, jos jokin muu oppimisratkaisu ei niitä pysty tarjoamaan.

%\subsection{Opettajan ja oppilaan tukeminen ohjelman tuottamien tietojen avulla}
Opettajan työtä ja oppilaan oppimista voidaan auttaa hyödyntämällä ohjelmiston tuottamaa dataa. Esimerkiksi 10Monkeys -sovellus muistaa mitä oppilas on aiemmin oppinut, ja oppilas näkee oman edistymisensä. Lisäksi opettaja voi seurata kaikkien oppilaiden edistymistä yksilöinä, sekä koko luokkana. Tästä on apua sen havaitsemissa, että kuka oppilaista on eniten avun tarpeessa, ja että kuinka nopeasti luokka kokonaisuutena etenee.
%\subsection{Maksuttomat sekä kaupalliset viihdepelit opetuksessa}
Myös muita kuin opetuspelejä voidaan käyttää hyödyksi opetuksessa. Oppilailla voidaan esimerkiksi peluuttaa Sims 2:sta ja sen jälkeen antaa tehtäväksi kirjoittaa pelin pohjalta kaunokirjallisia tekstejä~\parencite[][41-43]{laatua}.

%\subsection{Opettaminen pelejä kehittämällä}
Opettaa voi myös pelejä tekemällä. Esimerkiksi joissain kouluissa on pelikerhoja, joissa oppilaat tekevät itse opetuspelejä opettajien tarpeisiin. Pelejä voidaan tehdä esimerkiksi älytaulusovellusten tai esitysgrafiikan avulla. Hyvät ratkaisut saadaan otettua suoraan käyttöön opetuksessa. ~\parencite[][]{peleja}

%\subsection{Paikannuspohjaiset oppimisratkaisut}
Myös GPS-paikannukseen perustuvia oppimissovelluksia on alkanut ilmestyä. Näistä esimerkkejä ovat esimerkiksi Hei Joe- ja CityNomadi -sovellukset. Yleensä GPS-pohjaiset sovellukset ovat parhaimmillaan kun ympäristö pystytään liittämään mukaan mahdollisimman tukevasti. Esimerkiksi kaupungin historiaan liittyvien kohteiden kiertely ja kohteissa niihin liittyvien tehtävien tekeminen toimivat hyvin. ~\parencite[][]{heijoe} Näissä sovelluksissa voi olla valmiita reittejä, tai oppilaat voivat tehdä niitä myös itse. Reitit voivat olla esimerkiksi kaupungin patsaiden kiertelyä, ja kun saavutaan patsaalle, voidaan lukea patsaan esittämään henkilöön liittyvä tarina. Lisäksi esimerkiksi liikuntaan voi kannustaa monella tapaa. Moskovassa metroautomaatista saa lipun ilmaiseksi, jos tekee 30 jalkakyykkyä~\parencite[][]{kyykky}.
% https://citynomadi.com

%\subsection{Pilvipohjaiset toimisto-ohjelmistot opetuksessa}
Tietokoneilla esimerkiksi pilvipohjaiset toimisto-ohjelmistot mahdollistavat paljon asioita jotka aiemmin olivat jopa mahdottomia. Oppilaat voivat tehdä luontevasti samaa ryhmätyötä yhtä aikaa, jakaa keskeneräisetkin tuotokset reaaliaikaisesti koko luokan kanssa ja tehdä töitä  koulupäivän jälkeen ilman, että niitä tarvitsisi kantaa kotiin.
%TODO tähän lisää

\section{Laitteistoratkaisut}
%\subsection{Robotit opetuksessa}
Suosittu tapa käyttää robotteja opetuksessa ovat legorobotit. Ohjelmoitavia Legoja on käytetty opetuksessa Suomessa  Lego Control Labeista alkaen~\parencite[][s. 22-23]{skrolliRobot}. Control labeilla pystyttiin kasaamaan paikallaan olevia laitteita joita kontrolloitiin tietokoneelta. Varsinaiset Legorobotit tulivat käyttöön Lego Mindstorms-sarjan myötä. Sarjan robottien keskusyksikköön pystyttiin luomaan komentosarjoja yhdistelemällä Lego-palikan näköisiä osia mukana tulleessa ohjelmistossa. Niiden avulla robotin sai esimerkiksi reagoimaan valosensoriin tulevaan ärsykkeeseen liikkumalla. Legoroboteille on mahdollista kirjoittaa myös ohjelmia ohjelmointikielillä, ja tätä lähestymistapaa onkin käytetty esimerkiksi yliopistojen ohjelmoinnin opetuksessa~\parencite[][]{korppirobo, hesarobo}.

%Aktiivitaulut? älytykeistä maininta
%\subsection{Aktiivitaulut opetuksessa}
Aktiivitaulut ovat luokkahuonetta mullistava tekijä monessa mielessä. Niille opettajien on hyvin helppo tehdä interaktiivista materiaalia, ja esitysgrafiikkaan saa eloa, kun onkin aktiivinen toimija kuten liitutaululla, eikä vain dianvaihtaja. Oppilaille saa kehitettyä monenlaisia tehtäviä, ja taulut soveltuvat hyvin oppimissovellusten ja pelien käyttämiseen. Sovellusta käyttää vain yksi, mutta muiden on helppo seurata tapahtumia. Taululla saakin helposti aktivoitua oppilaita esimerkiksi erilaisilla järjestelytehtävillä.

%\subsection{Taulutietokoneet opetuksessa}
Aktiivitauluja seuraavat taulutietokoneet, joilla jokainen oppilas voi itsenäisesti tehdä samantyylisiä tehtäviä kuin älytaulullakin. Taulutietokoneet ovat parhaimmillaan omatoimisessa tehtävien teossa, sillä ne antavat välitöntä palautetta tehtävistä ja jokainen oppilas voi edetä omaan tahtiin. Tehtäviä ei tarvitse erikseen tarkistaa, koska taulutietokoneet kertovat, jos vastaus meni väärin. Myös yhteistyötä pystytään tehostamaan taulutietokoneiden tai tietokoneiden avulla.


\chapter{Yhteenveto}
%TODO tähän tekstiä
Innovatiiviset opetusmenetelmät eivät muuta koulutukseen liittyviä perusperiaatteita, mutta oikein käytettyinä niillä pystytään tarjoamaan uusia tapoja oppimiseen. Parhaimmillaan ne lisäävätkin oppimisen mielekkyyttä ja keventävät opettajien ja oppilaiden työtaakkaa.

\section{Tulokset}
Ennen Internetiä oli täysin mahdollista, että opettaja tiesi aiheista eniten ja pystyi tätä tietoa oppilailleen jakamaan. Nykyään kuitenkin monissa aineissa oppilaat pystyvät itse käymään erittäin hyvin jäsenneltyjä tietolähteitä lävitse varsin nopeasti, jolloin opettaja jää enemmän statistin rooliin. Opettajan rooliksi muotoutuukin entistä enemmän valmentajan rooli, jossa hän pyrkii ohjaamaan oppilaiden yksilöllistä oppimista mahdollisimman tehokkaasti ja auttamaan ymmärtämään juuri niissä kohdissa jossa kullakin on ongelmia, muiden asioiden jäädessä enemmän itseopiskelun varaan.

%Välitön palaute
Innovatiivisia opetusmenetelmiä käyttämällä esimerkiksi harjoitustehtävien tarkastuksen voi pitkälti jättää koneiden huoleksi, jolloin aiemmin yhteisesti tähän käytetty aika vapautuu oppilaiden neuvomiseen tehtävissä yksilöllisesti. Koneen tarkistaessa tehtävät välittömästi, oppilas saa palautteen ja osaa kysyä apua oikeaan ratkaisuun heti, sen sijaan, että apua saisi vasta kun havaitsee yhteisen tarkistuksen yhteydessä vastanneensa väärin. Usein tässä vaiheessa apu jää saamatta, koska ollaan jo siirtymässä seuraavaan aiheeseen tai “tehtävät on täytynyt olla vain tehtynä, muttei oikein”.

Myös opiskelun viihdyttävyyttä pystytään lisäämään varsin keveinkin teknisin ratkaisuin. Oppilaalle eteneminen voidaan esimerkiksi näyttää selkeästi kiinnostavalla grafiikalla, kun aiemmin se on näkynyt vain kirjan sivujen etenemisenä. Aiemminkin opettaja on tietysti voinut palkita oppilaita esimerkiksi tarroilla vihkoon, mutta koneen antama tunnustus vaikuttaisi erityisen toimivalta, mahdollisesti koska palaute on välittömämpää.

\section{Sähköisten opetusmenetelmien kehittämisessä huomioitavia asioita ja niiden tuoma lisäarvo}
% todo pura lausemuotoon.
Sähköisten menetelmien kehittämisessä huomiota tulee kiinnittää erityisesti menetelmien pedagogiseen puoleen sekä motivointiin. Pedagogisestikaan hyvä ratkaisu, jota on tylsä käyttää, tai jonka tekninen käytettävyys syö motivaation esimerkiksi viemällä 10 minuuttia jokaisen tunnin alusta sisäänkirjautumisilla, ei pidemmän päälle innosta opettajia saati oppilaita.

\subsection{Statistiikan hyödyntäminen}

Tilastojen avulla pystytään esimerkiksi kehittämään ja vertailemaan erilaisia sähköisiä materiaaleja toisiinsa. Tilastoista oleellisin on todennäköisesti se, että kuinka kauan oppilaat keskimäärin viihtyvät sähköisen materiaalin parissa ja palaavatko he sen pariin uudestaan vapaaehtoisesti. Pelialalla erääksi parhaista analytiikoista on havaittu se, että viihtyykö pelaaja pelin parissa 10 minuuttia tai enemmän ja palaako hän sen pariin vielä seuraavana päivänä. Opetusmateriaalien ollessa kyseessä tietysti voidaan analysoida myös esimerkiksi sitä, että kuinka nopeasti oppilaat käyvät opetussuunnitelman osa-alueita läpi.
 
 %kuinka paljon tilastoista saadaan irti, jenkit tilastoivat innokkaasti, miten tilastointia käytetään Suomessa?

 %Pelien statistiikkaa voi hyödyntää esimerkiksi seuraamalla koko luokan edistymistä, vertailla sitä aiempien vuosien luokkiin tai rinnakkaisluokkiin jne.

%\subsection{Hauskuuden ja oppimisen suhdeluku}
%Koulussa voi olla tylsäkin ohjelma, vapaaehtoisesti vapaa-ajalla käytettävien pitää painottaa hauskuutta oppimisen kustannuksella. Aikuisia ei voi pakottaa pelaamaan, lapset voi. Hyvä oppiminen vaatii sitä, että pelaajat tuottavat sisältöä, eivätkä pelkästään ole passiivisia kuluttajia.

 % http://www.academiccolab.org/resources/documents/Game%20Paper.pdf

%\subsection{Oppilaan taitotason mukaan skaalautuva haaste}
Opetussovelluksissa on mahdollista skaalata haastetta oppilaiden taitotason mukaan yksilöllisesti jokaiselle oppilaalle sopivaksi. Etuna on sekä se että sovellus tuntuu näin mielekkäämmältä ja oppilaan osaamistaso huomioidaan automaattisesti eikä anneta liian vaativia tehtäviä. Oppilaan toimet tallentamalla oppilas taas voi jälkikäteen tarkastella mitä tuli tehtyä, ja havaita virheensä ja paikat, joissa loisti. Sovellus voi myös palkita oppilasta monella tapaa. Sovellus voi esimerkiksi tarjota hyvästä menestyksestä erilaisia bonustasoja tai kunniamerkkejä. Oppilaalle voi myös näyttää edistymisen havainnollisesti. Tämä voi motivoida häntä etenemään nopeammin kohti lukukauden tavoitetta.~\parencite[][]{koulunArki, gamePaper}

Sisällön suhteen tulee kiinnittää huomiota myös kohderyhmän ikään. Ensimmäisen luokan oppilaille ei esimerkiksi kannata tehdä liian tekstipohjaista käyttöliittymää, koska lähtökohtaisesti kaikki eivät vielä osaa lukea. Lisäksi esimerkiksi väkivaltaa sisältävien sovellusten kohdalla tulee ikärajat ottaa huomioon.

%\subsection{Peleissä mahdollisuus viedä teoriaa käytäntöön}

Sovelluksissa voidaan viedä tehokkaasti teoriaa käytäntöön. Esimerkiksi simulaattoreissa pystytään opettamaan käytännön tehtäviä ilman, että on riskiä todellisista onnettomuuksista. Lisäksi olosuhteet pystytään simuloimaan, jolloin vaikkapa pimeäajoa voidaan opettaa jopa kesäisin. Autokoulut ovatkin siirtyneet simulaattoriopetukseen innokkaasti~\parencite[][]{simut}.

%\subsection{Reaaliaikainen palaute}
Myös reaaliaikaista palautetta pystytään antamaan perinteistä luokkahuoneopetusta tehokkaammin. Pelit ja sovellukset voivat reaaliaikaisesti tarkistaa oppilaan vastauksia ja kertoa, että ovatko ne oikein. Tämän jälkeen ne voivat myös tarjota mahdollisuutta vastata uudestaan. Välitön palaute varmistaa, ettei oppilas ehdi esimerkiksi oppia asiaa väärin tai pääse eteenpäin, jollei ole oikeasti ymmärtänyt mistä on kyse.

%\subsection{Palkitsevuus. Näkee edistymisensä}

%\subsection{Sisällön oltava kohderyhmälle sopivaa(iän osaamistason, väkivallan jne. suhteen)}

\section{Sähköisten menetelmien tuominen kouluihin}
%\subsection{Idean tuominen kouluun vertaisten kautta}
%Uudet työtavat siirtyvät muille opettajille tehokkaasti vertaisten kautta.  ~\parencite[][]{koulunArki} Myös tieto- ja viestintätekniikan pedagogista tukea kannattaa hyödyntää. ~\parencite[][s. 55-56]{kenttala}

%\subsection{Valmis opetussuunnitelmaan istuva paketti koko lukukauden oppimateriaalista}

Perinteiset kustantajien kouluille kauppaamat oppikirjat ja niiden oheismateriaalit kattavat koko lukukauden tai jopa koko lukuvuoden oppisisällön. Teknisten opetusratkaisujen tulisi pystyä tarjoamaan opettajille samaan tapaan riittävän laaja kokonaisuus eikä vain yksittäisiä sirpaleita, sillä sellaisesta materiaalista tuntien kokoaminen on työlästä. Jos sisällöt käsittelevät vain yhtä aihetta, niistä tulisi selkeästi käydä ilmi mitä osia opetussuunnitelmasta ne kattavat. Sähköisten ratkaisujen tulisi istua opetussuunnitelmaan, jotta niiden käyttö on helposti perusteltavissa. Ihannetilanteissa menetelmät myös antaisivat opettajalle lisää aikaa kahdenkeskiseen ohjaukseen, eivätkä veisi aikaa itse asian opetukselta verrattuna perinteisiin menetelmiin.

%\subsection{Opettajien ajanpuute perehtymiseen}
Opettajille tulisi järjestää riittävästi aikaa tietoteknisten valmiuksien kehittämiseen ja uusien työtapojen, laitteiden ja ohjelmistojen omaksumiseen osaksi arkea. Tulisi myös varmistaa, että uudet laitteet, ohjelmistot ja menetelmät oikeasti tulevat osaksi arkea, eikä niitä käytetä vain muutamalla tunnilla heti hankinnan jälkeen näön vuoksi. Opettajia olisi hyvä vähintäänkin kontaktoida uusien asioiden käyttöönoton jälkeen esimerkiksi viikosta neljään viikon välein ja muistuttaa ratkaisujen olemassaolosta, sekä kysyä niistä käyttökokemuksia.

%\subsection{Koulutukselle ei saisi olla tarvetta}
Jos ratkaisun käyttöön tarvitsee erikseen kouluttaa, niin haasteeksi tulee opettajien ajanpuute, sekä se, että koulutustarpeesta kilpailevat todennäköisesti muutkin ratkaisut ja asiat. Ratkaisujen tulisi olla riittävän yksinkertaisia, että ne pystyy omaksumaan helposti  itsenäisesti. Tämä ei kuitenkaan tarkoita sitä, etteikö koulutusta ja käyttöönoton jälkeistä käytön rutinoitumisen varmistusta tulisi järjestää mahdollisuuksien mukaan.

%\subsection{Oppimisen tehokkuus}
Jos ratkaisu on suunniteltu kouluun, siitä kuuluu oppia riittävän tehokkaasti, jotta aikaa asioiden omaksumiseen ei menisi merkittävästi enempää kuin opetussuunnitelmaa seuraamalla niihin on mahdollista käyttää. Kouluissa käytettävien ohjelmistojen ei tarvitse olla yhtä motivoivia kuin oppilaiden vapaa-ajalla käyttämien, koska niiden käytön ei tarvitse perustua vapaaehtoisuuteen. Samalla pystytään tuomaan oppimiseen enemmän painoa ja tehoa sisällöillä. Kotona vapaaehtoisesti pelattavat pelit ovat ensisijaisesti hauskoja. Kouluissa ne voivat olla ensisijaisesti opettavia. Parasta tietysti olisi, jos ohjelmistoilla saavutettaisiin flow-tila, johon pääsee peleissä haasteen ja viihteen yhdistyessä optimaalisesti~\parencite[][]{flow}. Oppimishaaste ja viihde tulisikin saada hyvään tasapainoon.


%TODO Hanki rdomilta paketti joka handlaa urlit
\printbibliography

\end{document}
