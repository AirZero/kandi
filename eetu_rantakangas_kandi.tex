\documentclass[utf8,bachelor]{gradu3}
\usepackage{graphicx} % kuvien mukaan ottamista varten
\usepackage{booktabs} % hyvä kauniiden taulukoiden tekemiseen
% HUOM! Tämän tulee olla viimeinen \usepackage koko dokumentissa!
\usepackage[bookmarksopen,bookmarksnumbered,linktocpage]{hyperref}
\addbibresource{eetu_rantakangas_kandi.bib} % Lähdetietokannan tiedostonimi
\begin{document}
\title{Innovatiiviset opetusmenetelmät}
\translatedtitle{Innovative teaching methods}
\studyline{Kaikki suuntautumisvaihtoehdot}
\avainsanat{Opetusmenetelmä, innovaatio}
\keywords{Teaching method, innovation}
\tiivistelma{%
  Tämä kirjoitelma on esimerkki siitä, kuinka
  {gradu3}-tutkielmapohjaa käytetään.  Se sisältää myös
  käyttöohjeet ja tutkielman rakennetta koskevia ohjeita.

  Tutkielman tiivistelmä on tyypillisesti lyhyt esitys, jossa
  kerrotaan tutkielman taustoista, tavoitteesta, tutkimusmenetelmistä,
  saavutetuista tuloksista, tulosten tulkinnasta ja johtopäätöksistä.
  Tiivistelmän tulee olla niin lyhyt, että se, englanninkielinen
  abstrakti ja muut metatiedot mahtuvat kaikki samalle sivulle.
}
\abstract{%
  This document is a sample {gradu3} thesis document class
  document.  It also functions as a user manual and supplies
  guidelines for structuring a thesis document.

  The English abstract of a thesis should usually say exactly the same
  things as the Finnish tiivistelmä.
}

\author{Eetu Rantakangas}
\contactinformation{\texttt{eetu.rantakangas@iki.fi}}
% jos useita tekijöitä, anna useampi \author-komento
\supervisor{-}
% jos useita ohjaajia, anna useampi \supervisor-komento

\maketitle

\preface
Tähän voit kirjoittaa tutkielmasi esipuheen. Tutkielmissa on
harvemmin esipuheita, mutta jos sen kirjoitat, pidä se lyhyenä
(enintään sivu).

Esipuheen tulisi kertoa ennemminkin tutkielmaprosessista kuin
tutkielman sisällöstä.  Esimerkiksi jos tutkielman aiheen valintaan
tai tekemiseen liittyy jokin erikoinen sattumus, voit siitä kertoa
esipuheessa.  Tapana esipuheessa on myös kiittää nimeltä mainiten
tärkeimpiä tutkielman tekemisessä auttaneita ihmisiä -- ainakin
ohjaajia, puolisoa ja lapsia.  (Yleensä perhe on auttanut vähintään
tukemalla ja kannustamalla.)

Esipuhe kannattaa kirjoittaa minä-muodossa. Tavanomaista on myös
allekirjoittaa se.

Jyväskylässä \today

\bigskip

Tutkielman tekijä

\begin{thetermlist}
\item[\TeX] Donald Knuthin 1977--1989 laatima eräajotyyppinen
  ladontajärjestelmä \parencite[ks.][]{knuth86:_texbook}. 
\item[\LaTeX] \TeX in \parencite[ks.][]{knuth86:_texbook} päälle
  rakennettu rakenteisten kirjoitelmien ladontaan tarkoitettu
  järjestelmä \parencite[ks.][]{lamport94:_latex}.  Siitä on nykyään
  käytössä versio \LaTeXe.
\end{thetermlist}

\mainmatter

\chapter{Johdanto}

Tutkimusongelma: Nykymuotoisen koulun uudistamisen tarpeesta
//TODO Tähän viittauksia erilaisiin koulujen uudistamistarveuutisiin sekä tutkimuksiin siitä kuinka koulut vastaavat nyky-yhteiskunnan tarpeita. Myös viittauksia kouluviihtyvyyteen
//Opetuskokonaisuudet muuttumassa laajemmiksi ainerajat kaatumassa

Suomen koulut ovat maailman huippua \parencite[][]{koulutHuippua}, mutta se ei tarkoita sitä, etteikö niitä voisi entisestään parantaa. Nykykouluissa oppilaat tuntevat olonsa kuormittuneiksi. \parencite [][]{oppilaidenKuormitus} Lisäksi oppilaiden kouluviihtyvyys on huonoa. Tähän liittyy myös esimerkiksi koulukiusaaminen. \parencite[][]{vakivalta} Opettajien tyytymättömyyttä taas lisää esimerkiksi palkattoman työn määrä. \parencite[][]{palkatonTyo}

Tutkimus käy läpi erilaisia innovatiivisiksi miellettäviä opetusmenetelmiä sekä esittelee sitä, miksi ne ovat innovatiivisia, ja kuinka nämä innovaatiot voivat auttaa opetuksen uudistuspyrkimyksissä. Lopuksi hahmotellaan suurempaa innovatiivisten menetelmien mahdollistamaa “perinteisen luokkahuonetilanteen” muutosta.

Suomi pärjää hyvin Pisa-vertailuissa, mutta kouluissa viihtyminen ei ole huipputasoa \parencite[][]{kouluViihtyvyys}, ja oppilaiden kuormituksen tunne kouluissa on yleistynyt. \parencite [][]{oppilaidenKuormitus} Tämän takia onkin seuraavaksi syytä kääntää katseet siihen, että miten viihtyvyyttä voitaisiin parantaa ja kuormituksen tunnetta vähentää. Esitellyt tavat eivät välttämättä ole suoria ratkaisuja siihen, ja ne ovat lähinnä yksittäisiä case-esimerkkejä. Niistä voi kuitenkin tulla merkittävä osa koulumaailman uudistustarpeen ratkaisukokonaisuutta. Ne auttavat hahmottamaan kaikkia niitä uusia tapoja ja mahdollisuuksia, joiden suuntaan toimintaa voidaan viedä. 

Voidaan myös kysyä, että mitä apua on uusista toimintatavoista, jos oppilaat kokevat koulutyön kuormittavaksi. Jos toimintatapoja tulee vain lisää, ja vanhat säilyvät rinnalla, koulunkäynnistä tulee helposti vaikeammin ennustettavaa ja opintojen työtapojen yhtäaikainen määrä voi olla omiaan aiheuttamaan hämmennystä.

Kuormittavuuden kannalta olisikin tärkeää, että uusia opetusmenetelmiä valittaessa huomioitaisiin niiden kuormittavuutta ja toisaalta myös kehitettäisiin ratkaisuja kuormittavuuteen. Jos oppilas esimerkiksi näkee läksyt ja deadlinet kerralla yhdestä paikkaa, voi hän helposti vain alkaa käydä listaa läpi ja merkkaamaan jo tehdyt, sen sijaan että muistelisi mitä kaikkea pitääkään tehdä ja musertuisi sekavan ja suuren läksykasan alle. Kun hahmottaa mitä pitää tehdä, ja voi merkata asian tehdyksi, niin työnteko on paljon mukavampaa. Myös läksyjen kerääntymistä voi pyrkiä välttämään samanlaisilla ratkaisuilla.

\chapter{Teoriaosa}
\section{Opetusmenetelmien määritelmä}
Innovaatiolla tarkoitetaan "idean tai keksinnön muuttamista tuotteeksi tai palveluksi josta asiakkaat ovat valmiita maksamaan." \parencite[][]{innovaatio} Toisin sanoen innovaatio tuottaa oleellista lisäarvoa. Oleellista lisäarvoa opetuksessa voi olla esimerkiksi oppilaan nopeampi oppiminen, opettajan työtehtävien automatisoituminen tai keveneminen, opettamisen tai oppimisen muuttaminen mielekkäämmäksi, varsinaisen asian oppimisen lisäksi sen rinnalla opittavat uudet taidot jne.

Tässä työssä tarkasteltavat opetusmenetelmät on rajattu ohjelmisto- ja laitteistoratkaisuihin. Kuitenkin myös esimerkiksi opetusmetodit voivat muuttua tietotekniikan avulla. Uusia lisäarvoa tuovia metodeita pystytään kehittämään myös täysin ilman tietotekniikkaa. Esimerkiksi flipped classroom voi toimia periaatteessa ilman tietotekniikkaakin. Tietotekniikkaa hyödynnettäessä oppilaat voivat kotona katsoa Youtube-videon seuraavan tunnin aiheesta ja tunnille tullessaan alkaa tehdä suoraan tehtäviä ilman opettajan johdanto-osuutta. Toisaalta oppilaat voisivat myös lukea johdanto-osuuden perinteisestä kirjasta. Flipped classroomissa ei kuitenkaan sinänsä tarvita mitään kohdistettua ohjelmistoratkaisua, joten esimerkiksi siihen ei tässä työssä sen syvemmin perehdytä. \parencite[][]{flipped} Myös esimerkiksi luokkahuoneen rakenne voidaan muuttaa perinteisestä pulpettimallisesta joksikin muuksi, jolloin voidaan käyttää erilaisia työtapoja kuin perinteisessä luokkahuoneessa.





\section{Innovatiivisten opetusmenetelmien käyttöönoton vaatimukset}

\subsection{valtionhallinnolliset ratkaisut}
Uusien ratkaisujen tuominen opetukseen ja koulumaailmaan ei ole täysin mutkatonta. Singaporessa koulujärjestelmää kehitetään tehokkaasti ja järjestelmällisesti. Singaporen malli mahdollistaa konkreettiset pitkän tähtäimen strategiat. Suomessa on enemmän hallinnon tasoja, mutta resursointi hallintoon on vähäisempää. Singaporessa esimerkiksi tietokoneet tarjotaan valtion puolelta kilpailutettuna, mutta kouluilla on myös mahdollisuus tehdä omat hankinnat, jos valtion tarjoamat vaihtoehdot eivät tyydytä tarpeita. Ylätason hallinto tukee esimerkiksi tällä tavalla kehittämistä voimakkaammin kuin Suomessa. \parencite[][]{koulunArki} 

//Ongelmat
Opettajat kokevat ajanpuutteen suurimmaksi esteeksi tietotekniikan hyödyntämiselle koulussa. Toisena syynä tulee koulun puutteellinen laitteisto ja kolmantena se, ettei oppilailla ole laitteita käytössä vapaa-ajallaan. \parencite[][]{sites}

% http://www.itviikko.fi/ihmiset-ja-ura/2010/08/02/opettajat-tunnustavat-puutteet-it-taidoissa/201010612/7

% TODO Mikseivät opettajat käytä enemmän tietotekniikkaa
\subsection{Opetusta uudistavat opettajat}
Kouluihin uusia opetuskäytänteitä vie opettajien "innovatiivinen osajoukko". Tämän opettajaosuuden aktivoiminen ja heidän taitojensa siirtäminen myös muille koulun opettajille voisikin olla merkittävässä roolissa uusien opetusmenetelmien jalkauttamisessa kouluihin.
\textit "Jokaisesta tutkimuskoulusta löytyi pieni joukko opettajia, jotka olivat omalla panoksellaan edistäneet innovatiivisten opetuskäytänteiden syntymistä. Nämä koulujen ns. innovaattoriopettajat olivat opetuskäytänteiltään pääosin samanlaisia kuin koulun muutkin opettajat. Erottavana tekijänä oli muita monipuolisempi valikoima erilaisia toimintatapoja. Innovatiiviset opettajat olivat monissa asioissa hyvin perinteisiä: opettajajohtoisia, vahvan struktuurin luokkaan luoneita sekä perinteisillä keinoilla oppilaitaan arvioivia opettajia. Näiden menetelmien lisäksi he käyttivät muita opettajia useammin oppilaslähtöisiä menetelmiä sekä tietotekniikkaa. Nämä opettajat myös yhdistelivät opetuksessaan innovatiivisia ja perinteisiä opetuskäytänteitä. He olivat toiminnallaan tuoneet koulukulttuuriin jotain uutta perinteisen toimintamallin ulkopuolelta. Innovaattoriopettajat joutuivat kuitenkin kohtaamaan usein paineita koko koulun uudistamisesta sekä muutosvastarintaa muilta opettajilta. He jäivät usein työssään myös liian yksin. Tällaiset opetusta uudistavat opettajat tarvitsevat tukea koulun rehtorilta ja muulta työyhteisöltä. He tarvitsevat yhteisön arvostusta sekä ulkopuolisia näkökulmia opetuskäytänteidensä arvioimiseen sekä levittämiseen muualle kouluun. Vastuu innovatiivisen opetuskäytänteen leviämisestä ei saisi jäädä yksin innovaattoriopettajien vastuulle." \parencite[][]{koulunArki}

\subsection{Yhteistyön lisääminen}
Ontariossa Kanadassa on saatu nostettua alueen koulujen oppimistuloksia lisäämällä opettajien välistä yhteistyötä. Kun opettajilla on hyvät keskusteluyhteydet kollegoiden kanssa, ongelmat ratkeavat nopeammin ja tehokkaammin kuin yksin pähkäillessä. Lisäksi kouluissa on sitouduttu yhteisiin periaatteisiin:

“Ontarion koulut ovat sitoutuneet muutamiin yhteisiin periaatteisiin. Ne ovat oppilaslähtöisyys, avoimuus, läpinäkyvyys, luovat opetusmetodit sekä oppilaiden rohkaiseminen ja kannustaminen niin, että myös epäonnistumiset sallitaan. Yhteiset tavoitteet ovat esillä luokkien seinillä.” \parencite[][]{KanadanMalli} 

Jos oppilaat voidaan sitouttaa koulun sääntöihin kuten “tupakointi kielletty” miksei opettajiakin voitaisi sitouttaa yhteisiin tavoitteisiin?

Opetuksen yhteistyötä saadaan myös lisättyä esimerkiksi aineiden yhteisillä tunneilla. Esimerkiksi tietotekniikkaa voidaan hyödyntää vaikka maantiedon tunnilla tai liikuntaa yhdistää historiantuntiin. Tietysti myös esimerkiksi biologian ja fysiikan yhteäviä kohtia pystytään hyödyntämään.

\subsection{Oppimateriaalien jakaminen ja verkosta löytyvien hyödyntäminen}
Oppimateriaalien jako opettajien kesken on myös nykyään paljon helpompaa kuin ennen. Kollegat pystyvät vähentämään toistensa työtaakkaa suunnittelemalla materiaaleja yhdessä, jakamalla niitä keskenään sekä hyödyntämällä jo netistä valmiiksi löytyvää materiaalia.


\subsection{Johdon rooli innovatiivisten opetusmenetelmien käyttöönotossa}
Johto pystyy vaikuttamaan innovatiivisten opetusmenetelmien käyttöönottoon erityisesti kannustamalla opettajia. Lisäksi kannattaa järjestää koulutusta laitteisiin ja ohjelmistoihin, ja valita koulutuksessa parhaiten pärjännyt osuus henkilökunnasta toimimaan jatkossa muiden apuna laitteiden ja ohjelmistojen käytössä. Johto pystyy myös vaikuttamaan mahdollisuuksiin opettaa eri tavoilla esimerkiksi lisäämällä kaksoistuntien määrää, jolloin opettajat voivat viedä oppilaat ulos tai tehdä muita aikaa enemmän vaativia asioita. 
 
 Rehtorien panos innovatiivisten opettajien taitojen levittämisessä muille opettajille voisi olla merkittävä, toisaalta ainakaan vielä vuonna 2006 rehtorit eivät nähneet tietotekniikkaa erityisen merkittävänä osana opetusta ja sen kehittämistä. \parencite[][]{koulunArki}

\subsection{Opetuksen pedagoginen tvt-tuki}
Opetuksen pedagogisella tvt-tuella tarkoitetaan tukea, joka ei auta opettajaa niinkään teknisesti, vaan pääasiassa auttaa opettajaa käyttämään opetuksessaan ja liittämään opetukseensa tieto- ja viestintäteknologiaa hyödyntävää materiaalia. (ritva-liisa). Tällä hetkellä on ainakin kaksi erilaista toimivaksi havaittua lähestymistapaa tähän ongelmaan. Toinen on tietotekniikan opettajan tarjoama pedagoginen tvt-tuki ja toinen taas muiden opettajien tarjoama. Muiden opettajien tarjoama tuki saadaan aikaiseksi esimerkiksi kouluttamalla kaikki opettajat käyttämään tietoteknistä ratkaisua tai välinettä, ja näistä opettajista perataan ne, jotka omaksuivat käytön parhaiten, eli noin 1/10. Tämän opettajajoukon tehtävänä on jatkossa tukea muita tvt-välineiden käytössä. Tämä toimii esimerkiksi älytaulun tai muiden laitteiden kannalta varsin hyvin, mutta pitkällä aikavälillä ei välttämättä auta opettajia omaksumaan uusia välineitä ja ratkaisuja, koska niistä paras asiantuntemus on yleensä tietotekniikan opettajalla tai koulun ulkopuolisilla asiantuntijoilla. Joka tapauksessa tällainen ratkaisu on toteutettavissa koulun henkilöstömäärää lisäämättä, joten muiden opettajien vertaistuen hyödyntäminen on erittäin suositeltavaa.

\subsection{Flipped classroom}
Matematiikan puolella \parencite[][]{maot} ehkä eniten puhetta viime aikoina herättänyt opetusmenetelmä on ns. flipped classroom.\parencite[][]{flipped} Oppilaille ei opeteta asioita koulussa, vaan oppilaat opettelevat asiat kotona, ja koulussa keskitytään laskemiseen. Jos jotain ei ole ymmärretty, niin opettaja neuvoo. Tällä tavoin opettajalle jää enemmän aikaa oppilaiden henkilökohtaiseen ohjaukseen kuin mallissa, jossa opettaja aina esittelisi luokan edessä uuden asian. Vastaavasti opettajan on mahdollista ottaa koko ryhmän kanssa esiin vain kaikkein haastavimmiksi osoittautuneet asiat. Käytännössä jokainen opiskelija opiskelee omaan tahtiinsa, ja opettaja auttaa tarvittaessa. Tämä myös rikkoo perinteisiä kurssi- ja vuosiluokkarajoja ja opiskelijoiden oma tahti jatkuu läpi koko koulun. Tavoitteena kaikilla on kuitenkin täyttää opetussuunnitelman asettamat vaatimukset. Samalla saadaan hyöty irti mahdollisista sähköisistä ohjelmista, jotka antavat sopivantasoista laskettavaa ja korjaavat heti jos vastaus meni väärin.
% http://maot.fi/oppimisymparisto/oppimisympariston-perusidea/

\subsection{Opettaminen pelejä kehittämällä}
Opettaa voi myös pelejä tekemällä. Esimerkiksi joissain kouluissa on pelikerhoja, joissa oppilaat tekevät itse opetuspelejä opettajien tarpeisiin. Pelejä voidaan tehdä esimerkiksi älytaulusovellusten tai powerpointin avulla. Hyvät ratkaisut saadaan otettua suoraan käyttöön opetuksessa. \parencite[][]{peleja}


% http://www.katsomo.fi/?progId=378128

% http://yle.fi/elavaarkisto/artikkelit/tietokoneet_koulujen_opetuksen_tukena_105423.html#media=105426

\section{Asioita joita sähköisten opetusmenetelmien kehittämisessä kannattaa huomioida ja tapoja joilla ne tuovat lisäarvoa}
% todo pura lausemuotoon.
1. Statistiikan hyödyntäminen.
Pelien statistiikkaa voi hyödyntää esimerkiksi seuraamalla koko luokan edistymistä, vertailla sitä aiempien vuosien luokkiin tai rinnakkaisluokkiin jne.
2. Hauskuuden ja oppimisen suhdeluku
	Koulussa voi olla tylsäkin ohjelma, vapaaehtoisesti vapaa-ajalla käytettävien pitää painottaa hauskuutta oppimisen kustannuksella.
Aikuisia ei voi pakottaa pelaamaan, lapset voi. Hyvä oppiminen vaatii sitä, että pelaajat tuottavat, eivätkä pelkästään ole passiivisia kuluttajia. % http://www.academiccolab.org/resources/documents/Game%20Paper.pdf

3. Oppilaan taitotason mukaan skaalautuva haaste \parencite[][]{koulunArki} \parencite[][13-14]{gamePaper}
Peleissä on myös mahdollista skaalata haastetta oppilaiden taitotason mukaan yksilöllisesti jokaiselle oppilaalle sopivaksi. Etuna on sekä se että peli tuntuu näin mielekkäämmältä sekä paremmalta ja oppilaan osaamistaso huomioidaan automaattisesti eikä anneta liian vaativia tehtäviä.


4. Oppilaan toimien tallennus - oppilas voi jälkikäteen tarkastella mitä tuli tehtyä, ja havaita virheensä ja paikat joissa loisti.
5. Palkitsevuus. Näkee edistymisensä.
6. Sisällön oltava kohderyhmälle sopivaa(iän osaamistason, väkivallan jne. suhteen) 
7. Peleissä mahdollisuus viedä teoriaa käytäntöön

8. Reaaliaikainen palaute.
Pelit ja ohjelmistot voivat reaaliaikaisesti tarkistaa oppilaan vastauksia ja kertoa, että ovatko ne oikein. Tämän jälkeen ne voivat myös tarjota mahdollisuutta vastata uudestaan. Välitön palaute varmistaa, ettei oppilas ehdi esimerkiksi oppia asiaa väärin tai pääse eteenpäin, jollei ole oikeasti ymmärtänyt mistä on kyse.

\section{Sähköisten menetelmien tuominen kouluihin}
1. Idean tuominen kouluun vertaisten kautta
Uudet työtavat siirtyvät muille opettajille tehokkaasti vertaisten kautta. \parencite[][]{koulunArki}
2. Valmis paketti koko lukukauden oppimateriaalista
Perinteiset kustantajien kouluille kauppaamat oppikirjat ja niiden oheismateriaalit kattavat koko lukukauden tai jopa koko lukuvuoden oppisisällön. Myös teknisten opetusratkaisujen tulisi pystyä tarjoamaan opettajille riittävän laaja kokonaisuus eikä vain yksittäisiä sirpaleita, sillä sellaisesta materiaalista tuntien kokoaminen on työlästä. Jos sisällöt käsittelevät vain yhtä aihetta, niistä tulisi selkeästi käydä ilmi mitä kaikkea ne kattavat.
3. Opettajien ajanpuute perehtymiseen.

4. Koulutukselle ei saisi olla tarvetta.
Jos ratkaisun käyttöön tarvitsee erikseen kouluttaa, niin haasteeksi tulee opettajien ajanpuute, sekä se, että koulutustarpeesta kilpailevat todennäköisesti muutkin ratkaisut ja asiat.
5. Opetussuunnitelmaan sisällöllisesti istuva.
Sähköisten ratkaisujen tulisi istua opetussuunnitelmaan, jotta niiden käyttö on helposti perusteltavissa.
6. Jos kouluun, täytyy olla tehokas tapa oppia, jos kotiin ei niin väliä.

\chapter{Innovatiiviset menetelmät}

\section{Materiaalin ja kokemusten luokittelu ja jakaminen}
Opetushallituksen sivuilta löytyy hyvät käytännöt-osio, jossa on listattuna monenmoisia opettajien opetuksessaan hyödyntämiä ideoita. \parencite[][]{hyvatkaytannot} Kuka tahansa opettaja pystyy lähettämään sivustolle omia ideoitaan. Tällaiset kokoavat sivustot nousevatkin yhä tärkeämpään asemaan materiaalin määrän koko ajan kasvaessa. Samantapaisia kokoavia sivustoja on myös esimerkiksi opetusmateriaalien ja pelien löytämiseen ja jakamiseen. Esimerkiksi graphite.orgista löytyy tagiperustaista sähköisen oppimateriaalin jaottelua. lemill.net puolestaan tarjoaa mahdollisuuden jakaa omaa tai muiden oppimateriaalia sekä merkitä materiaalin käyttöoikeudet.
% http://www.graphite.org
% http://lemill.net
% http://www.peda.net/veraja/konnevesi/lukio/ophhanke2010/pelit

\section{Robotit opetuksessa}
Suosittu tapa käyttää robotteja opetuksessa ovat legorobotit. Ohjelmoitavia Legoja on käytetty opetuksessa Suomessa  Lego Control Labeista alkaen. \parencite[][]{skrolliRobot} Control labeilla pystyttiin kasaamaan paikallaan olevia laitteita joita kontrolloitiin tietokoneelta. Varsinaiset Legorobotit tulivat käyttöön Lego Mindstorms-sarjan myötä. Sarjan robottien keskusyksikköön pystyttiin luomaan komentosarjoja yhdistelemällä Lego-palikan näköisiä osia mukana tulleessa ohjelmistossa. Niiden avulla robotin sai esimerkiksi reagoimaan valosensoriin tulevaan ärsykkeeseen liikkumalla. Legoroboteille on mahdollista kirjoittaa myös ohjelmia ohjelmointikielillä, ja tätä lähestymistapaa onkin käytetty esimerkiksi yliopistojen ohjelmoinnin opetuksessa. \parencite[][]{korppirobo} \parencite[][]{hesarobo}

\section{Oppimispelit ja sovellukset}
Oppimispelit ja sovellukset pystytään jakamaan kahteen pääluokkaan: ensisijaisesti viihdyttävät ja ensisijaisesti opettavat. Jälkimmäisten hyödyllisyyttä pystyy ehkä parhaiten arvioimaan viihdyttävyyden määrällä ja oppimisen tehokkuudella. Jos sovelluksen käyttö on vähintään yhtä viihdyttävää kuin tunnilla olo, ja siitä oppii yhtä hyvin tai paremmin asiat kuin perinteisellä oppitunnilla, sovellus on yksinkertaisesti hyvä. Näiden asioiden mittaaminen vaatii kuitenkin suhteellisen laajat koejärjestelyt, joten yleensä perstuntuma on paras mittari.

Hyvä esimerkki tehokkaasta oppimissovelluksesta on http://memrise.com. Sivuston tarkoitus on opettaa kielten sanastoja. Perinteiseen sanakirjaan verrattuna sivustossa on valtavasti etuja. Se esimerkiksi tietää, kuinka nopeasti ja millä tavoin opit parhaiten. Tämä perustuu käyttäjädatan hyödyntämiseen - tiedetään, monta toistoa sana tarvitsee, että ihmiset sen keskimäärin oppivat. Lisäksi tiedetään, että kuinka monta kertaa kannattaa antaa tehtävä, jossa kysytään: “Mikä sana on näistä neljästä?”, ja kuinka monta kertaa kannattaa kysyä, että: “Miten sana kirjoitetaan?”. Etuna sanakirjaan on myös se, että sivusto myös tarjoaa sanoihin liittyviä muistisääntöjä kuvina ja lauseina sekä antaa sanoista ääntämisesimerkit. Nämä helpottavat muistamista entisestään, koska oppiminen on monikanavaisempaa. Sivusto myös pyytää “kastelemaan” opitut sanat tietyin väliajoin, etteivät ne unohdu.

Sivustolla on myös jonkin verran pelillisyyttä. Eniten kyseistä kielikurssia viikon aikana käyneet näkyvät ranking-listoilla top10:ssä ja sanojen opiskelussa käytetään kukkavertauksia. Ensin kasvi istutetaan, jonka jälkeen kerätään sato. Tässä vaiheessa sana on opittu. Tämän jälkeen täytyy kasvia vielä kastella, jotta se ei kuole, eli opittu ei unohdu. 

Memrisestäkin toisaalta puuttuu kaikki opittua soveltava. Sieltä ei löydy lukemista joka liittyisi juuri opittuihin sanoihin tai esimerkiksi mahdollisuutta tuottaa tekstiä tai keskustella. Myöskään kielioppia ei sieltä opi. Näiden asioiden opettaminen jääkin opettajan ja perinteisen kieltenopetuksen vastuulle, jos jokin muu oppimisratkaisu ei niitä pysty tarjoamaan.
Opettajan ja oppilaan tukeminen ohjelman tuottamien tietojen avulla
Opettajan ja oppilaan oppimista voidaan auttaa hyödyntämällä ohjelmiston tuottamaa dataa. Esimerkiksi 10Monkeys -sovellus muistaa mitä oppilas on aiemmin oppinut, ja oppilas näkee oman edistymisensä. Lisäksi opettaja voi seurata kaikkien oppilaiden edistymistä yksilöinä, sekä koko luokkana. Tästä on apua sen havaitsemissa, että kuka oppilaista on eniten avun tarpeessa, ja että kuinka nopeasti luokka kokonaisuutena etenee.
Maksuttomat sekä kaupalliset viihdepelit opetuksessa
Myös muita kuin opetuspelejä voidaan käyttää hyödyksi opetuksessa. Oppilailla voidaan esimerkiksi peluuttaa Sims 2:sta ja sen jälkeen antaa tehtäväksi kirjoittaa pelin pohjalta kaunokirjallisia tekstejä.
% http://www.oph.fi/julkaisut/2012/laatua_e_oppimateriaaleihin


\section{Paikannuspohjaiset oppimisratkaisut}
Myös GPS-paikannukseen perustuvia oppimisratkaisuja on alkanut ilmestyä. Näistä esimerkkejä ovat esimerkiksi Hei Joe ja CityNomadi -ympäristöt. Yleensä GPS-pohjaiset ratkaisut ovat parhaimmillaan kun ympäristö pystytään liittämään mukaan mahdollisimman tukevasti. Esimerkiksi kaupungin historiaan liittyvien kohteiden kiertely ja kohteissa niihin liittyvien tehtävien tekeminen toimivat hyvin.
% http://www.opettaja.fi/cs/Satellite?c=Page&pagename=OpettajaLehti%2FPage%2Fjuttusivu&cid=1351276519632&juttuID=1355755427510
% https://citynomadi.com

Opetusta voi viedä arkiympäristöihin esimerkiksi gps-pohjaisilla oppimisratkaisuilla, kuten CityNomadi tai Hei Joe -sovelluksilla. Näissä sovelluksissa voi olla valmiita reittejä, tai oppilaat voivat tehdä niitä myös itse. Reitit voivat olla esimerkiksi kaupungin patsaiden kiertelyä, ja kun saavutaan patsaalle, voidaan lukea patsaan esittämään henkilöön liittyvä tarina. Myös esimerkiksi liikuntaan voi kannustaa monella tapaa. Moskovassa metroautomaatista saa lipun ilmaiseksi, jos tekee 30 jalkakyykkyä.

% http://www.hs.fi/ulkomaat/a1383884611093?jako=e5772b9fe7f9a22f030e592a36c5b9d7&ref=og-url
% https://citynomadi.com
% http://www.opettaja.fi/cs/Satellite?c=Page&pagename=OpettajaLehti%2FPage%2Fjuttusivu&cid=1351276519632&juttuID=1355755427510

\section{Älytaulut opetuksessa}
Älytaulut ovat luokkahuonetta mullistava tekijä monessa mielessä. Niille opettajien on hyvin helppo tehdä interaktiivista materiaalia, ja myös powerpoint-esityksiin saa eloa, kun onkin aktiivinen toimija kuten liitutaululla, eikä vain dianvaihtaja. Myös oppilaille saa kehitettyä monenlaisia tehtäviä, ja taulut soveltuvat hyvin myös oppimissovellusten ja pelien käyttämiseen. Sovellusta käyttää vain yksi, mutta muiden on helppo seurata tapahtumia. Taululla saa myös helposti aktivoitua oppilaita esimerkiksi erilaisilla järjestelytehtävillä.

\section{Padit opetuksessa}
Tauluja seuraavat padit, joilla jokainen oppilas voi itsenäisesti tehdä samantyylisiä tehtäviä kuin taulullakin. Padit ovat parhaimmillaan omatoimisessa tehtävien teossa, sillä ne antavat välitöntä palautetta tehtävistä ja jokainen oppilas voi edetä omaan tahtiin, tehtäviä ei tarvitse erikseen tarkistaa, koska padit kertovat jos vastaus meni väärin. Myös yhteistyötä pystytään padien tai tietokoneiden avulla.
Pilvipohjaiset toimisto-ohjelmistot opetuksessa
Tietokoneilla esimerkiksi pilvipohjaiset toimisto-ohjelmistot mahdollistavat paljon asoiita jotka aiemmin olivat jopa mahdottomia. Oppilaat voivat tehdä luontevasti samaa ryhmätyötä yhtä aikaa, jakaa keskeneräisetkin tuotokset reaaliaikaisesti koko luokan kanssa ja tehdä töitä myös koulupäivän jälkeen ilman, että niitä tarvitsisi kantaa kotiin.


\section{Tulokset}
// Ennen Internetiä oli täysin mahdollista, että opettaja tiesi aiheista eniten ja pystyi myös tätä tietoa oppilailleen jakamaan. Nykyään kuitenkin monissa aineissa oppilaat pystyvät itse käymään erittäin hyvin jäsenneltyjä tietolähteitä lävitse varsin nopeasti, jolloin opettaja jää enemmän statistin rooliin. Opettajan rooliksi muotoutuukin entistä enemmän valmentajan rooli, jossa hän pyrkii ohjaamaan oppilaiden yksilöllistä oppimista mahdollisimman tehokkaasti ja auttamaan ymmärtämään juuri niissä kohdissa jossa kullakin on ongelmia, muiden asioiden jäädessä enemmän itse opiskelun varaan.

Innovatiivisia opetusmenetelmiä käyttämällä esimerkiksi harjoitustehtävien tarkastuksen voi pitkälti jättää koneiden huoleksi, jolloin aiemmin yhteisesti tähän käytetty aika vapautuu oppilaiden neuvomiseen tehtävissä yksilöllisesti. Lisäksi koneen tarkistaessa tehtävät heti kun ne on tehnyt, oppilas saa palautteen ja osaa kysyä apua oikeaan ratkaisuun heti, sen sijaan, että apua saisi vasta sitten kun havaitsee yhteisen tarkistuksen yhteydessä vastanneensa väärin. Usein tässä vaiheessa apu jää myös saamatta, koska ollaan jo siirtymässä seuraavaan aiheeseen tai “tehtävät on täytynyt olla vain tehtynä, muttei oikein”.

Myös opiskelun viihdyttävyyttä pystytään lisäämään varsin keveinkin teknisin ratkaisuin. Oppilaalle eteneminen voidaan esimerkiksi näyttää selkeästi kiinnostavalla grafiikalla, kun aiemmin se on näkynyt vain kirjan sivujen etenemisenä. Aiemminkin opettaja on tietysti voinut palkita oppilaita esimerkiksi tarroilla vihkoon, mutta koneen antama tunnustus vaikuttaisi erityisen toimivalta, mahdollisesti koska palaute on välittömämpää.





\end{document}
